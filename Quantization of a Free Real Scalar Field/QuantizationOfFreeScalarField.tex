\documentclass{article}

\usepackage{amsmath}
\usepackage{framed}
\usepackage{mathrsfs}

\usepackage[left = 1cm,right = 1cm, top = 2cm]{geometry}

\begin{document}

    \title{How to Quantize a Real Scalar Field (Free)}
    \maketitle

    The Lagrangian Density should be:

    \begin{equation}
        \mathcal{L} = \frac{1}{2} \eta^{\mu \nu} \partial_{\mu} \phi \partial_{\nu} \phi - \frac{1}{2} m^{2} \phi^{2}
    \end{equation}

    The equation of motion is:

    \begin{equation}
        \frac{\partial \mathcal{L}}{\partial \phi} - \partial{\mu} \Big( \frac{\partial \mathcal{L}}{\partial (\partial_{\mu} \phi)} \Big) = 0
    \end{equation}

    \begin{equation}
        \partial^{\mu} \partial_{\mu} \phi + m^{2} \phi = 0
    \end{equation}

    This equation has the following general solution in terms of plane-waves:

    \begin{equation}
        k \cdot x = k_{\mu} x^{\mu}
    \end{equation}

    \begin{equation}
        \omega = k_{0} = \sqrt{\vec{k} \cdot \vec{k} + m^{2}}
    \end{equation}

    \begin{equation}
        \phi (\vec{x}, t) = \frac{1}{(2 \pi)^3} \int \Big[ a(\vec{k}) e^{-i k \cdot x} + a^{\dag}(\vec{k}) e^{i k \cdot x} \Big] \frac{d^3k}{2 \omega}
    \end{equation}

    Where making the second Fourier coefficient the Hermitian Conjugate of $a(\vec{k})$ ensures that the scalar field is invariant under
    complex conjugation, and is therefore real. Also the integration measure, $\frac{d^3 k}{2 \omega}$, is Lorentz Invariant. The 
    conjugate field has the following value:

    \begin{equation}
        \pi (\vec{x}, t) = \frac{\partial \mathcal{L}}{\partial (\partial_{\mu} \phi)} = \dot{\phi}
    \end{equation}

    \begin{equation}
        \pi (\vec{x}, t) = \frac{1}{(2 \pi)^3} \int (- i \omega) \Big[ a(\vec{k}) e^{-i k \cdot x} - a^{\dag}(\vec{k}) e^{i k \cdot x} \Big] \frac{d^3k}{2 \omega}
    \end{equation}

    Now let's quantize. Generalizing from Quantum Mechanics:

    \begin{equation}
        \begin{aligned}[]
            [\phi (\vec{x}, t),\pi (\vec{y}, t)] & = i \delta^{3}(\vec{x} - \vec{y}) \\
            [\phi (\vec{x}, t), \phi (\vec{y}, t)] & = 0 \\
            [\pi (\vec{x}, t), \pi (\vec{y}, t)] & = 0
        \end{aligned}
    \end{equation}

    In order to satisfy the commutation relation, $\phi (\vec{x}, t)$ and $\pi (\vec{x}, t)$ must be taken to be operators instead of numbers.
    This can only be the case if the Fourier coefficients, $a(\vec{k})$ and $a^{\dag}(\vec{k})$, in the expression above for $\phi (\vec{x}, t)$
    and $\pi (\vec{x}, t)$ are taken to be operators that satisfy particular commutation relations. The above commutation relations and expressions
    for $\phi (\vec{x}, t)$ and $\pi (\vec{x}, t)$ can be used to deduce commutation relations for $a(\vec{k})$ and $a^{\dag}(\vec{k})$ that are
    consistent with the commutation relations of $\phi (\vec{x}, t)$ and $\pi (\vec{x}, t)$. This proceeds as follows:

    \begin{equation}
        \begin{aligned}
            \int e^{i k' \cdot x} \phi (\vec{x}, t) d^{3}k  = \frac{1}{(2 \pi)^3} \int (- i \omega) \Big[ a(\vec{k}) e^{-i (k' - k) \cdot x} - a^{\dag}(\vec{k}) e^{i (k' + k) \cdot x} \Big] \frac{d^3k}{2 \omega} \\
            = \frac{1}{(2 \pi)^3} \int (- i \omega) \Big[ a(\vec{k}) e^{-i (\omega' + \omega) \cdot x} e^{-i (k' + k) \cdot x} - a^{\dag}(\vec{k}) e^{i (\omega' - \omega) \cdot x} e^{i (k' - k) \cdot x} \Big] \frac{d^3k}{2 \omega}
        \end{aligned}
    \end{equation}

    \begin{equation}
        \begin{aligned}
            \frac{1}{(2 \pi)^3} \int e^{-i (k' - k) \cdot x} d^{3}x = \delta^{3} (\vec{k'} - \vec{k}) \\
            \frac{1}{(2 \pi)^3} \int e^{-i (k' + k) \cdot x} d^{3}x = \delta^{3} (\vec{k'} + \vec{k})
        \end{aligned}
    \end{equation}

    \begin{equation} 
        \begin{aligned}
            \int e^{i k' \cdot x} \phi (\vec{x}, t) d^{3}k & = \frac{1}{(2 \pi)^3} \int (- i \omega) \Big[ a(\vec{k}) e^{-i (\omega' + \omega) \cdot x} e^{-i (k' + k) \cdot x} - a^{\dag}(\vec{k}) e^{i (\omega' - \omega) \cdot x} e^{i (k' - k) \cdot x} \Big] \frac{d^3k}{2 \omega} \\
            & = \frac{1}{2 \omega'} \big[ a(\vec{k'}) + a(-\vec{k'}) e^{i 2 \omega' t} \big]
        \end{aligned}
    \end{equation}

    Now that the integration over $\vec{k}$ is done, there is no reason to keep the prime marker on the other
    momentum variable, so the final answer is:

    \begin{equation}
        \int e^{i k' \cdot x} \phi (\vec{x}, t) d^{3}k = \frac{1}{2 \omega'} \big[ a(\vec{k'}) + a(-\vec{k'}) e^{i 2 \omega' t} \big]
    \end{equation}

    Now the same calculation can be done for $\pi (\vec{x}, t)$:

    \begin{equation}
        \begin{aligned}
            \int e^{i k' \cdot x} \pi (\vec{x}, t) d^{3}k  = \frac{1}{(2 \pi)^3} \int (- i \omega) \Big[ a(\vec{k}) e^{-i (k' - k) \cdot x} - a^{\dag}(\vec{k}) e^{i (k' + k) \cdot x} \Big] \frac{d^3k}{2 \omega} \\
            = \frac{1}{(2 \pi)^3} \int (- i \omega) \Big[ a(\vec{k}) e^{-i (\omega' + \omega) \cdot x} e^{-i (k' + k) \cdot x} - a^{\dag}(\vec{k}) e^{i (\omega' - \omega) \cdot x} e^{i (k' - k) \cdot x} \Big] \frac{d^3k}{2 \omega}
        \end{aligned}
    \end{equation}

    \begin{equation}
        \begin{aligned}
            \frac{1}{(2 \pi)^3} \int e^{-i (k' - k) \cdot x} d^{3}x = \delta^{3} (\vec{k'} - \vec{k}) \\
            \frac{1}{(2 \pi)^3} \int e^{-i (k' + k) \cdot x} d^{3}x = \delta^{3} (\vec{k'} + \vec{k})
        \end{aligned}
    \end{equation}

    \begin{equation} 
        \begin{aligned}
            \int e^{i k' \cdot x} \pi (\vec{x}, t) d^{3}k & = \frac{1}{(2 \pi)^3} \int (- i \omega) \Big[ a(\vec{k}) e^{-i (\omega' + \omega) \cdot x} e^{-i (k' + k) \cdot x} - a^{\dag}(\vec{k}) e^{i (\omega' - \omega) \cdot x} e^{i (k' - k) \cdot x} \Big] \frac{d^3k}{2 \omega} \\
            & = \frac{i}{2} \big[ a(\vec{k'}) + a(-\vec{k'}) e^{i 2 \omega' t} \big]
        \end{aligned}
    \end{equation}

    Again, now that the integration over $\vec{k}$ is done, there is no reason to keep the prime marker on the other momentum variable, so the final answer is:

    \begin{equation}
        \int e^{i k' \cdot x} \pi (\vec{x}, t) d^{3}k = \frac{i}{2} \big[ a(\vec{k'}) + a(-\vec{k'}) e^{i 2 \omega' t} \big]
    \end{equation}

    With these two relations, it is now easy to write $a(\vec{k})$ and $a^{\dag}(\vec{k})$ in terms of $\phi (\vec{x}, t)$ and $\pi (\vec{x}, t)$. One can immediately see that:

    \begin{framed}
        \begin{equation}
            a(\vec{k}) = \int e^{i k x} [i \pi (\vec{x}, t) + \omega \phi(\vec{x}, t)] d^{3} x
        \end{equation}
    \end{framed}

    \begin{framed}
        \begin{equation}
            a^{\dag}(\vec{k}) = \int e^{- i k x} [- i \pi (\vec{x}, t)+ \omega \phi(\vec{x}, t)] d^{3} x
        \end{equation}
    \end{framed}

    The $a(\vec{k})$ and $a^{\dag}(\vec{k})$ operators are called the harmonic oscillator raising and lowering operators because of their similar properties to
    those in the famous harmonic oscillator Schrodinger Equation problem. Now, we can commute these to figure out their commutation relations. Let's start with:

    \begin{equation}
        \begin{aligned}[]
        [a(\vec{k}), a^{\dag}(\vec{k})] = \int e^{i k \cdot x - i k' \cdot x'} [(i \pi (\vec{x}, t) + \omega \phi(\vec{x}, t))(- i \pi (\vec{x}, t)+ \omega \phi(\vec{x}, t)) \\ 
        - (- i \pi (\vec{x}, t)+ \omega \phi(\vec{x}, t))(i \pi (\vec{x}, t) + \omega \phi(\vec{x}, t))] d^{3} x d^{3} x'
        \end{aligned}
    \end{equation}

    Now applying $[\phi(\vec{x}, t), \phi(\vec{y}, t)] = 0$ and $[\pi(\vec{x}, t), \pi(\vec{y}, t)] = 0$

    \begin{equation}
        [a(\vec{k}), a^{\dag}(\vec{k})] = \int e^{i k \cdot x - i k' \cdot x'} [i \omega [\pi(\vec{x}, t), \phi(\vec{y}, t)] - i \omega [\phi(\vec{x}, t), \pi(\vec{y}, t)]] d^{3} x d^{3} x'
    \end{equation}

    Now applying $[\phi(\vec{x}, t), \pi(\vec{y}, t)] = \delta^{3} (\vec{x} - \vec{y})$

    \begin{equation}
        \begin{aligned}[]
            [a(\vec{k}), a^{\dag}(\vec{k})] = \int e^{i k \cdot x - i k' \cdot x'} \delta^{3} (\vec{x} - \vec{y}) d^{3} x d^{3} x' = 2 \omega \int e^{i (k - k') \cdot x} d^{3} x \\
            = 2 \omega \int e^{i (\omega - \omega') \cdot x} e^{i (k - k') \cdot x} d^{3} x = (2 \pi)^{3} 2 \omega e^{i (\omega - \omega') \cdot x} \delta^{3} (\vec{k} - \vec{k'}) \\
            = (2 \pi)^{3} 2 \omega \delta^{3} (\vec{x} - \vec{y})
        \end{aligned}
    \end{equation}

    \begin{framed}
        \begin{equation}
            [a(\vec{k}), a^{\dag}(\vec{k})] = (2 \pi)^{3} 2 \omega \delta^{3} (\vec{x} - \vec{y})
        \end{equation}
    \end{framed}

    One can also show that $a(\vec{k})$ and $a^{\dag}(\vec{k})$ commute with themselves:

    \begin{equation}
        \begin{aligned}[]
            [a(\vec{k}), a(\vec{k})] = \int e^{i k \cdot x - i k' \cdot x'} [i \omega [\pi(\vec{x}, t), \phi(\vec{y}, t)] - i \omega [\phi(\vec{x}, t), \pi(\vec{y}, t)]] d^{3} x d^{3} x' \\
            = \int e^{i k \cdot x - i k' \cdot x'} [- i \omega \delta^{3} (\vec{x} - \vec{x'}) + i \omega \delta^{3} (\vec{x} - \vec{x'})] d^{3} x d^{3} x' = 0
        \end{aligned}
    \end{equation}

    \begin{equation}
        \begin{aligned}[]
            [a^{\dag}(\vec{k}), a^{\dag}(\vec{k})] = \int e^{i k \cdot x - i k' \cdot x'} [i \omega [\pi(\vec{x}, t), \phi(\vec{y}, t)] - i \omega [\phi(\vec{x}, t), \pi(\vec{y}, t)]] d^{3} x d^{3} x' \\
            = \int e^{i k \cdot x - i k' \cdot x'} [- i \omega \delta^{3} (\vec{x} - \vec{x'}) + i \omega \delta^{3} (\vec{x} - \vec{x'})] d^{3} x d^{3} x' = 0
        \end{aligned}
    \end{equation}

    With these commutation relations, the energy states of a system can be worked out. First, the Hamiltonian must be computed, then the above established commutation relations for $a(\vec{k})$ and $a^{\dag}(\vec{k})$ can 
    be used to work out what the energy eigenstates are. From Noether's theorem, we have that the stress energy momentum tensor is:
    
    \begin{equation}
        T^{\mu \nu} = \frac{\partial \mathcal{L}}{\partial (\partial_{0} \phi)} \partial_{0} \phi - \mathcal{L} = \frac{1}{2} (\pi^{2} + \nabla \phi \cdot \nabla \phi + m^{2} \phi^{2})
    \end{equation}

    \begin{equation}
        H = \int \mathcal{H} d^{3} x = \frac{1}{2} \int (\pi^{2} + \nabla \phi \cdot \nabla \phi + m^{2} \phi^{2}) d^{3} x
    \end{equation}
    
    Where

    \begin{equation}
        \phi (\vec{x}, t) = \frac{1}{(2 \pi)^3} \int \Big[ a(\vec{k}) e^{-i k \cdot x} + a^{\dag}(\vec{k}) e^{i k \cdot x} \Big] \frac{d^3k}{2 \omega}
    \end{equation}

    \begin{equation}
        \pi (\vec{x}, t) = \frac{1}{(2 \pi)^3} \int (- i \omega) \Big[ a(\vec{k}) e^{-i k \cdot x} - a^{\dag}(\vec{k}) e^{i k \cdot x} \Big] \frac{d^3k}{2 \omega}
    \end{equation}

    \begin{equation}
        \nabla \phi (\vec{x}, t) = \frac{1}{(2 \pi)^3} (i \vec{k}) \int \Big[ a(\vec{k}) e^{-i k \cdot x} + a^{\dag}(\vec{k}) e^{i k \cdot x} \Big] \frac{d^3k}{2 \omega}
    \end{equation}

    Therefore:

    \begin{equation}
        \begin{aligned}[]
            H = \frac{1}{2 (2 \pi)^3 (2 \pi)^3} \int \frac{1}{2} \{ - \omega \omega' [a(\vec{k}) e^{-i k \cdot x} - a^{\dag}(\vec{k}) e^{i k \cdot x}][a(\vec{k}) e^{-i k \cdot x} - a^{\dag}(\vec{k}) e^{i k \cdot x}] \\
            - \vec{k} \cdot \vec{k'} [a(\vec{k}) e^{-i k \cdot x} - a^{\dag}(\vec{k}) e^{i k \cdot x}][a(\vec{k}) e^{-i k \cdot x} - a^{\dag}(\vec{k}) e^{i k \cdot x}] \\
            + m^{2} [a(\vec{k}) e^{-i k \cdot x} + a^{\dag}(\vec{k}) e^{i k \cdot x}][a(\vec{k}) e^{-i k \cdot x} + a^{\dag}(\vec{k}) e^{i k \cdot x}] \}  d^{3} k d^{3} k' d^{3} x
        \end{aligned}
    \end{equation}

    \begin{equation}
        \begin{aligned}[]
            H = \frac{1}{2 (2 \pi)^3 (2 \pi)^3} \int \{ (- \omega \omega' - \vec{k} \cdot \vec{k'}) [a(\vec{k}) e^{-i k \cdot x} - a^{\dag}(\vec{k}) e^{i k \cdot x}][a(\vec{k}) e^{-i k \cdot x} - a^{\dag}(\vec{k}) e^{i k \cdot x}] \\
            + m^{2} [a(\vec{k}) e^{-i k \cdot x} + a^{\dag}(\vec{k}) e^{i k \cdot x}][a(\vec{k}) e^{-i k \cdot x} + a^{\dag}(\vec{k}) e^{i k \cdot x}] \} d^{3} k d^{3} k' d^{3} x
        \end{aligned}
    \end{equation}

    \begin{equation}
        \begin{aligned}[]
            \int e^{i (k - k') \cdot x} d^{3} x = 2 \omega \int e^{i (\omega - \omega') \cdot x} e^{i (k - k') \cdot x} d^{3} x =  \\
            = (2 \pi)^{3} 2 \omega e^{i (\omega - \omega') \cdot x} \delta^{3} (\vec{k} - \vec{k'}) \\
            = (2 \pi)^3 \delta^{3} (\vec{k} - \vec{k'})
        \end{aligned}
    \end{equation}

    \begin{equation}
        \begin{aligned}[]
            \int e^{i (k + k') \cdot x} d^{3} x = 2 \omega \int e^{i (\omega - \omega') \cdot x} e^{i (k - k') \cdot x} d^{3} x =  \\
            = (2 \pi)^{3} 2 \omega e^{i (\omega + \omega') \cdot x} \delta^{3} (\vec{k} + \vec{k'}) \\
            = e^{2 i \omega t} (2 \pi)^3 \delta^{3} (\vec{k} + \vec{k'})
        \end{aligned}
    \end{equation}

    \begin{equation}
        \begin{aligned}[]
            H = \frac{1}{2 (2 \pi)^3 (2 \pi)^3} \int \{ (- \omega \omega' - \vec{k} \cdot \vec{k'}) [a(\vec{k}) a(\vec{k'}) (2 \pi)^{3} e^{-2 i \omega t} \delta^{3}(\vec{k} + \vec{k'}) \\
            - a(\vec{k}) a^{\dag}(\vec{k'}) (2 \pi)^3 \delta^{3} (\vec{k} - \vec{k'}) - a^{\dag}(\vec{k}) a(\vec{k'}) (2 \pi)^3 \delta^{3} (\vec{k} - \vec{k'}) \\
            + a^{\dag}(\vec{k}) a^{\dag}(\vec{k'}) e^{- 2 i \omega t} \delta^{3} (\vec{k} + \vec{k'}) ]\\
            + m^2 [ a(\vec{k}) a(\vec{k'}) e^{- 2 i \omega t} \delta^{3} (\vec{k} + \vec{k'}) + a(\vec{k}) a^{\dag}(\vec{k'}) \delta^{3} (\vec{k} - \vec{k'}) \\
            + a^{\dag}(\vec{k}) a(\vec{k'}) \delta^{3} (\vec{k} - \vec{k'}) ] \} d^{3} k d^{3} k' d^{3} x
        \end{aligned}
    \end{equation}

    Remembering the properties of the delta function, and that $\omega = k_{0} = \sqrt{\vec{k} \cdot \vec{k} + m^{2}}$, one finds that:

    \begin{equation}
        H = \frac{1}{2 (2 \pi)^3 (2 \pi)^3} \int \delta^{3} (\vec{k} - \vec{k'}) \{ (- \omega \omega' - \vec{k} \cdot \vec{k'}) [a(\vec{k}) a^{\dag}(\vec{k'}) + a^{\dag}(\vec{k}) a(\vec{k'})] \}  d^{3} k d^{3} k' d^{3} x
    \end{equation}

    Now doing the k' Integration:

    \begin{equation}
        \begin{aligned}[]
            H = \frac{(2 \pi)^3}{2 (2 \pi)^3 (2 \pi)^3} \int \{ (- \omega^{2} - \vec{k} \cdot \vec{k'}) [a(\vec{k}) a^{\dag}(\vec{k'}) + a^{\dag}(\vec{k}) a(\vec{k'})] \}  d^{3} k d^{3} x \\
            = \frac{(2 \pi)^3}{2 (2 \pi)^3 (2 \pi)^3} \int \{ 2 \omega^{2} [a(\vec{k}) a^{\dag}(\vec{k'}) + a^{\dag}(\vec{k}) a(\vec{k'})] \}  d^{3} k d^{3} x
        \end{aligned}
    \end{equation}

    \begin{equation}
        H = \frac{1}{2} \int \omega [a(\vec{k}) a^{\dag}(\vec{k}) + a^{\dag}(\vec{k}) a(\vec{k})] \frac{d^{3} k}{(2 \pi)^{3} 2 \omega}
    \end{equation}

    Next Applying the $[a(\vec{k}), a^{\dag}(\vec{k})] = 2 \omega \delta^{3} (\vec{k'} - \vec{k})$:

    \begin{equation}
        H = \frac{1}{2} \int \omega [2 a(\vec{k}) a^{\dag}(\vec{k}) + 2 \omega \delta^{3} (0)] \frac{d^{3} k}{(2 \pi)^{3} 2 \omega}
    \end{equation}

    Infinite (and finite) shifts at the zero point of energy scale cannot be detected,
    so the last term cannot be ignored. Doing this gives the normal ordered Hamiltonian:
    
    \begin{framed}
        \begin{equation}
            H_{n} = \int \omega a(\vec{k}) a^{\dag}(\vec{k}) \frac{d^{3} k}{(2 \pi)^{3} 2 \omega}
        \end{equation}
    \end{framed}

    Normal ordering is formally defined as positioning all $a^{\dag}(\vec{k})$ operators (later we will see that these are creation operators)
    to the left and all $a(\vec{k})$ (these will turn out to be annihilation operators) to the right. Doing this yields more sensible quantum
    physical expressions. Given that the ordering of classical fields is arbitrary, and the original Hamiltonian was computed by inserting 
    quantized fields (field operators) into classical formulas, one can see that the ordering of quantum fields is Ambiguous. Taking the normal
    ordered Hamiltonian as the correct one can be viewed as simply taking the only available ordering that doesn't yield nonsense. Also notice
    that $H_{n}$ is Hermitian. With the normal ordered Hamiltonian, the spectrum of energy states of the theory can be computed.
    
    \begin{equation}
        H_{n} | E \rangle = E | E \rangle
    \end{equation}

    \begin{equation}
        \begin{aligned}
            H_{n} a(\vec{k}) | E \rangle = (E - \omega) a(\vec{k}) | E \rangle = \int \omega' a^{\dag} (\vec{k'}) a (\vec{k'}) a (\vec{k}) | E \rangle \frac{d^{3} k'}{(2 \pi)^{3} 2 \omega'} \\
            = \int \omega' a^{\dag} (\vec{k'}) a (\vec{k}) a (\vec{k'}) | E \rangle \frac{d^{3} k'}{(2 \pi)^{3} 2 \omega'} \\
            = \int \omega' a^{\dag} (\vec{k'}) \big[a (\vec{k}) a^{\dag} (\vec{k'}) - [a (\vec{k}), a^{\dag} (\vec{k})] \big] a (\vec{k'}) | E \rangle \frac{d^{3} k'}{(2 \pi)^{3} 2 \omega'} \\
            = \int \omega' a^{\dag} (\vec{k'}) \big[a (\vec{k}) a^{\dag} (\vec{k'}) - (2 \pi)^3 2 \omega \delta^{3} (\vec{k} - \vec{k'}) \big] a (\vec{k'}) | E \rangle \frac{d^{3} k'}{(2 \pi)^{3} 2 \omega'} \\
            = a (\vec{k}) \int \omega' a^{\dag} (\vec{k}) a (\vec{k'}) | E \rangle \frac{d^{3} k'}{(2 \pi)^{3} 2 \omega'} - \omega a (\vec{k}) = a (\vec{k}) H_{n} | E \rangle - \omega a (\vec{k}) | E \rangle \\
            = a (\vec{k}) E | E \rangle - \omega a (\vec{k}) | E \rangle = H_{n} a(\vec{k}) | E \rangle = (E - \omega) a(\vec{k}) | E \rangle
        \end{aligned}
    \end{equation}

    \begin{framed}
        \begin{equation}
            H_{n} a(\vec{k}) | E \rangle = (E - \omega) a(\vec{k}) | E \rangle
        \end{equation}
    \end{framed}

    Therefor, we see that $a(\vec{k})$ is a lowering or annihilation operator (as mentioned above). It removes a quantum of energy that it acts on.
    Now, we can perform a similar calculation to find the meaning of $a^{\dag}(\vec{k})$

    \begin{equation}
        \begin{aligned}
            H_{n} a^{\dag}(\vec{k}) | E \rangle = \int \omega' a^{\dag} (\vec{k'}) a (\vec{k'}) a (\vec{k}) | E \rangle \frac{d^{3} k'}{(2 \pi)^{3} 2 \omega'} \\
            = \int \omega' a^{\dag} (\vec{k'}) \big[a (\vec{k}) a^{\dag} (\vec{k'}) - [a (\vec{k}), a^{\dag} (\vec{k'})] \big] a (\vec{k'}) | E \rangle \frac{d^{3} k'}{(2 \pi)^{3} 2 \omega'} \\
            = \int \omega' a^{\dag} (\vec{k'}) \big[a (\vec{k}) a^{\dag} (\vec{k'}) + [a^{\dag} (\vec{k'}), a (\vec{k})] \big] a (\vec{k'}) | E \rangle \frac{d^{3} k'}{(2 \pi)^{3} 2 \omega'} \\
            = \int \omega' a^{\dag} (\vec{k'}) \big[a (\vec{k}) a^{\dag} (\vec{k'}) + (2 \pi)^3 2 \omega \delta^{3} (\vec{k} - \vec{k'}) \big] a (\vec{k'}) | E \rangle \frac{d^{3} k'}{(2 \pi)^{3} 2 \omega'} \\
            = a (\vec{k}) \int \omega' a^{\dag} (\vec{k}) a (\vec{k'}) | E \rangle \frac{d^{3} k'}{(2 \pi)^{3} 2 \omega'} + \omega a (\vec{k}) \\
            = a (\vec{k}) H_{n} | E \rangle + \omega a (\vec{k}) | E \rangle \\
            = (E + \omega) a^{\dag}(\vec{k}) | E \rangle
        \end{aligned}
    \end{equation}

    \begin{framed}
        \begin{equation}
            H_{n} a^{\dag}(\vec{k}) | E \rangle = (E + \omega) a^{\dag}(\vec{k}) | E \rangle
        \end{equation}
    \end{framed}

    Therefor, we see that $a^{\dag}(\vec{k})$ is a raising or creation operator (as mentioned above). It adds to the quantum of energy to the state
    it acts on. From here, a few important realizations can be made. First, because $H_{n}$ is an integral over $a(\vec{k}) a^{\dag}(\vec{k})$ its
    expectation values are non-negative:

    \begin{equation}
        \langle \psi | H_{n} | \rangle \geq 0
    \end{equation}

    With this in mind, because $a(\vec{k})$ lowers the value of the eigenvalue by one quantum, there must be a lowest state that gets zeroed by 
    $a^{\dag}(\vec{k})$:

    \begin{equation}
        a(\vec{k}) | 0 \rangle = 0
    \end{equation}

    This ground state will be taken as the definition of the vacuum, because it has no quanta of energy left to be annihilated. One can then interpret
    these energy quanta to be particles (in this case, spin zero massive particles because we are quantizing a massive scalar field theory). With these
    interpretation, one can see that field quantization has had the following effect. At a given frequency, only discrete energy levels can be allowed,
    and they can correspond to different integer numbers of particles (field quanta) being present at a particular frequency. Basically, the states of
    the theory simply correspond to a vacuum containing some number of massive spin zero particles with various frequencies.

    One can act on the vacuum state with the creation operator to populate it with particles. A one particle state can be written the following way:

    \begin{equation}
        a^{\dag}(\vec{k}) | 0 \rangle = | \vec{k} \rangle
    \end{equation}

    The complete spectrum of energy eigenstates can be written as follows:

    \begin{equation}
        \prod_{i = 1}^{K} \frac{[a^{\dag}(\vec{k}_{i})]^{n (\vec{k}_{i})}}{\sqrt{n (\vec{k}_{i})!}} | 0 \rangle = | n(\vec{k}_{1})...n(\vec{k}_{K}) \rangle
    \end{equation}

    Because creation operators commute with each other, these field quanta obey Bose-Einstein statistics, and are bosons. The zero point on the energy scale is
    set to such that the energy of such vacuum has a value of zero. As a consequence:

    \begin{equation}
        H | \vec{k} \rangle = \omega | \vec{k} \rangle
    \end{equation}

    Also, it is instructive to compute the momentum:

    \begin{equation}
        \mathcal{P}^{i} = T^{0 i} = \frac{\partial \mathcal{L}}{\partial (\partial_0 \phi)} \partial^{i} \phi = - \pi \nabla \phi
    \end{equation}

    \begin{equation}
        \mathcal{P}^{i} = \int \mathcal{P}^{i} d^{3} x = - \int \pi \nabla \phi d^{3} x
    \end{equation}

    \begin{equation}
        \pi (\vec{x}, t) = \frac{1}{(2 \pi)^3} \int (- i \omega) \Big[ a(\vec{k}) e^{-i k \cdot x} - a^{\dag}(\vec{k}) e^{i k \cdot x} \Big] \frac{d^3k}{2 \omega}
    \end{equation}

    \begin{equation}
        \nabla \phi (\vec{x}, t) = \frac{1}{(2 \pi)^3} (i \vec{k}) \int \Big[ a(\vec{k}) e^{-i k \cdot x} + a^{\dag}(\vec{k}) e^{i k \cdot x} \Big] \frac{d^3k}{2 \omega}
    \end{equation}

    \begin{equation}
        \begin{aligned}[]
            \int e^{i (k - k') \cdot x} d^{3} = 2 \omega \int e^{i (\omega - \omega') \cdot x} e^{i (k - k') \cdot x} d^{3} x  \\
            = (2 \pi)^{3} 2 \omega e^{i (\omega - \omega') \cdot x} \delta^{3} (\vec{k} - \vec{k'}) \\
            = (2 \pi)^3 \delta^{3} (\vec{k} - \vec{k'})
        \end{aligned}
    \end{equation}

    \begin{equation}
        \begin{aligned}[]
            \int e^{i (k + k') \cdot x} d^{3} = 2 \omega \int e^{i (\omega - \omega') \cdot x} e^{i (k - k') \cdot x} d^{3} x  \\
            = (2 \pi)^{3} 2 \omega e^{i (\omega + \omega') \cdot x} \delta^{3} (\vec{k} + \vec{k'}) \\
            = e^{2 i \omega t} (2 \pi)^3 \delta^{3} (\vec{k} + \vec{k'})
        \end{aligned}
    \end{equation}

    \begin{equation}
        \begin{aligned}[]
            \vec{P} = \int \omega' \vec{k} [ a (\vec{k'}) a (\vec{k}) e^{- 2 i \omega t} (2 \pi)^{3} \delta^{3} (\vec{k} + \vec{k'}) \\
            - a (\vec{k}) a^{\dag} (\vec{k}) (2 \pi)^{3} \delta(\vec{k} - \vec{k'}) - a^{\dag} (\vec{k}) a (\vec{k}) (2 \pi)^{3} \delta(\vec{k} - \vec{k'}) \\
            + a^{\dag} (\vec{k}) a^{\dag} (\vec{k}) (2 \pi)^{3} \delta(\vec{k} + \vec{k'})] \frac{d^3k}{2 \omega} \frac{d^3k'}{2 \omega'}
        \end{aligned}
    \end{equation}

    \begin{equation}
        \vec{P} = - \frac{1}{2} \int \vec{k} [a (-\vec{k}) a (\vec{k}) e^{-2 i \omega t} - a (\vec{k}) a^{\dag} (\vec{k}) a^{\dag} (\vec{k}) a (\vec{k}) a^{\dag} (-\vec{k}) a^{\dag} (\vec{k}) e^{2 i \omega t}] \frac{d^3k}{2 \omega}
    \end{equation}

    \begin{equation}
        \begin{aligned}[]
            \vec{P} = \frac{1}{2} \int \vec{k} [a (\vec{k}) a (-\vec{k}) e^{- 2 i \omega t} + a^{\dag} (\vec{k}) a^{\dag} (-\vec{k}) e^{- 2 i \omega t}] \frac{d^3k}{2 \omega} \\
            + \frac{1}{2} \int \vec{k} [a (\vec{k}) a^{\dag} (\vec{k}) + a^{\dag} (\vec{k}) a (\vec{k})] \frac{d^3k}{2 \omega}
        \end{aligned}
    \end{equation}

    The first integral has an integrand that is odd in $k^i$, so it vanishes, therefor:

    \begin{equation}
        \vec{P} = \frac{1}{2} \int [a (\vec{k}) a^{\dag} (\vec{k}) + a^{\dag} (\vec{k}) a (\vec{k})] \frac{d^3k}{2 \omega}
    \end{equation}

    \begin{equation}
        \vec{P} = \int \vec{k} a^{\dag} (\vec{k}) a (\vec{k}) \frac{d^3k}{2 \omega} + \frac{1}{2} \int \vec{k} [a (\vec{k}), a^{\dag} (\vec{k})] \frac{d^3k}{2 \omega}
    \end{equation}

    The second integral has an integrand that is odd in $k^i$, so it vanishes, therefor:

    \begin{framed}
        \begin{equation}
                \vec{P} = = \int k a^{\dag}(\vec{k}) a(\vec{k}) \frac{d^{3} k}{(2 \pi)^{3} 2 \omega}
        \end{equation}
    \end{framed}

    With this formula for the momentum operator, one finds:

    \begin{equation}
        \vec{P} | \vec{k} \rangle = \vec{k} | \vec{k} \rangle
    \end{equation}

    The momentum operator therefor just gives the momentum when applied to an energy eigenstate of the system.
    With this done, there is only one thing left to do, and that is to introduce the number operator:

    \begin{equation}
        N (\vec{k}) = a^{\dag} (\vec{k}) a (\vec{k})
    \end{equation}

    This operator has a following interesting property:

    \begin{equation}
        N (k_i) | N (k_1)...N (k_i)...N (k_K) \rangle = n (k_i) | N (k_1)...N (k_i)...N (k_K) \rangle
    \end{equation}

    This is easy to show by commuting $a (\vec{k})$ to the right until it annihilates into a vacuum. The operator
    has eigenvalues equal to the nubmer of quanta of momentum $\vec{k}$ present in the energy eigenstate to which
    it is applied. Its eigenvalues are called occupation numbers. It turns out that the energy and momentum 
    operators appear a lot more intuitive if thought about in terms of this number operator. Specifically, we have:

    \begin{equation}
        H_{n} = \int \omega a(\vec{k}) a^{\dag}(\vec{k}) \frac{d^{3} k}{(2 \pi)^{3} 2 \omega} = \int \omega N(\vec{k}) \frac{d^{3} k}{(2 \pi)^{3} 2 \omega}
    \end{equation}

    \begin{equation}
        \vec{P} = \int k a^{\dag}(\vec{k}) a(\vec{k}) \frac{d^{3} k}{(2 \pi)^{3} 2 \omega} = \int \vec{k} N(\vec{k}) \frac{d^{3} k}{(2 \pi)^{3} 2 \omega}
    \end{equation}

    From this, one can see very intuitively, that the Hamiltonian, and the momentum operators really do just add up
    the energy and momentum respectively of all of the quanta present in a given state.

\end{document}