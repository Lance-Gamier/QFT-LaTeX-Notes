\documentclass[a4]{article}

\usepackage{amsmath}
\usepackage{framed}
\usepackage{mathrsfs}

\usepackage[left = 1cm,right = 1cm, top = 2cm]{geometry}

\begin{document}

    \title{How to Quantize a Dirac Fermion Field}
    \maketitle

    The Dirac Lagrangian density in Natural Units is:

    \begin{equation}
        \mathcal{L} = \bar{\psi} i \gamma^{\mu} \partial_{\mu} \psi - m \bar{\psi} \psi
    \end{equation}

    Where $\psi$ is the complex field, and $\bar{\psi} = \psi^{\dagger} \gamma^{0}$. $\psi$ and $\bar{\psi}$ must be
    treated as independent fields in the Lagrangian Formalism. The Equations of Motions are:
    
    \begin{equation}
        \begin{aligned}[]
            \frac{\partial \mathcal{L}}{\partial \bar{\psi}} - \partial_{\mu} \bigg( \frac{\partial \mathcal{L}}{\partial (\partial_{\mu} \bar{\psi})} \bigg) = 0 \quad \rightarrow \quad i \gamma^{\mu} \partial_{\mu} \psi - m \psi = 0 \\
            \frac{\partial \mathcal{L}}{\partial \psi} - \partial_{\mu} \bigg( \frac{\partial \mathcal{L}}{\partial (\partial_{\mu} \psi)} \bigg) = 0 \quad \rightarrow \quad i \gamma^{\mu} \partial_{\mu} \bar{\psi} - m \bar{\psi} = 0
        \end{aligned}
    \end{equation}

    The cannonical conjugate are as follows:

    \begin{equation}
        \pi_\psi = \partial (\partial_{\mu} \bar{\psi}) = i \psi^{\dagger} \hspace*{1.5cm} \pi_{\bar{\psi}} = \partial (\partial_{\mu} \psi) = 0
    \end{equation}

    For the Hamiltonian density, we have:

    \begin{equation}
        \mathcal{H} = \sum_{i} \pi_{i} \dot{\psi}_{i} - \mathcal{L} = \bar{\psi} i \gamma^{0} \partial_{0} \psi - \bar{\psi} i \gamma^{\mu} \partial_{\mu} \psi + m \bar{\psi} \psi = - \bar{\psi} i \gamma^{\mu} \partial_{\mu} \psi + m \bar{\psi} \psi
    \end{equation}

    However, we can simplify this with the Dirac Equation:

    \begin{equation}
        i \gamma^{\mu} \partial_{\mu} \psi - m \psi \quad \rightarrow \quad i \gamma^{0} \partial_{0} \psi = - \bar{\psi} i \gamma^{\mu} \partial_{\mu} \psi + m \bar{\psi} \psi
    \end{equation}

    Therefore:

    \begin{equation}
      \mathcal{H} = - \bar{\psi} i \gamma^{\mu} \partial_{\mu} \psi + m \bar{\psi} \psi = i \gamma^{0} \partial_{0} = i psi^{\dagger} \partial_{0} \psi
    \end{equation}

    \begin{equation}
      H = \int \mathcal{H} d^{3} x = \int (i \psi^{\dagger} \partial_{0} \psi) d^{3} x
    \end{equation}

    The momentum density and momentum are:

    \begin{equation}
        \mathcal{P}^{i} = - \bar{\psi} i \gamma^{i} \partial_{i} \psi = - \psi^{\dagger} \nabla \psi
    \end{equation}

    \begin{equation}
        \vec{P} = \int \vec{\mathcal{P}} d^{3} x = - \int (\psi^{\dagger} \nabla \psi) d^{3} x
    \end{equation}

    Global U(1) invariance gives the following conserved Noether current:

    \begin{equation}
        J^{\mu} = \bar{\psi} \gamma^{\mu} \psi \hspace*{1.0cm} \partial_{\mu} J^{\mu} = 0 
    \end{equation}

    The associated conserved charge is:

    \begin{equation}
        \mathcal{Q} = \int \bar{\psi} \gamma^{0} \psi d^{3} x = \int \psi^{\dagger} \psi d^{3} x
    \end{equation}

    The charge can be associated with the electric charge. It's operator will be used to tell the charge of
    the various quantum states.

    \begin{center}
        \begin{tabular}{c c c}
            $\psi^{+} (x, t) = u_{s} e^{- i p \cdot x}$ & $u_{s} (p) = \frac{\gamma^{\mu} p_{\mu} + m}{\sqrt{2m(m + \omega)}} u_{s}$ & $u_{1} = \left(\begin{array}{c} 1 \\ 0 \\ 0 \\ 0 \end{array}\right)$ $u_{2} = \left(\begin{array}{c} 0 \\ 1 \\ 0 \\ 0 \end{array}\right)$ \\
            $\psi^{-} (x, t) = v_{s} e^{ i p \cdot x}$ & $v_{s} (p) = \frac{- \gamma^{\mu} p_{\mu} + m}{\sqrt{2m(m + \omega)}} v_{s}$ & $v_{1} = \left(\begin{array}{c} 0 \\ 0 \\ 1 \\ 0 \end{array}\right)$ $v_{2} = \left(\begin{array}{c} 0 \\ 0 \\ 0 \\ 1 \end{array}\right)$
        \end{tabular}
    \end{center}

    Where we have that:

    \begin{equation}
        \omega = k_{0} = \sqrt{\vec{k} \cdot \vec{k} + m^{2}} 
    \end{equation}

    \begin{equation}
        \left(\begin{array}{cc} I & 0 \\ 0 & I \end{array}\right) \hspace*{1cm} \left(\begin{array}{cc} 0 & \sigma^{i} \\ - \sigma^{i} & 0 \end{array}\right)
    \end{equation}

    One can construct the general solution to the Dirac Equation by taking an arbitrary linear combination of
    these plane wave solutions (using standard normalization):

    \begin{equation}
        \psi (x, t) = \sum_{s} \int \frac{m}{\omega} [e^{- i p \cdot x} + e^{i p \cdot x}] \frac{d^{3} p}{(2 \pi)^3}
    \end{equation}

    \begin{equation}
        \psi^{\dagger} (x, t) = \sum_{s} \int \frac{m}{\omega} [e^{- i p \cdot x} + e^{i p \cdot x}] \frac{d^{3} p}{(2 \pi)^3}
    \end{equation}

    \begin{equation}
        \bar{\psi} (x, t) = \sum_{s} \int \frac{m}{\omega} [e^{- i p \cdot x} + e^{i p \cdot x}] \frac{d^{3} p}{(2 \pi)^3}
    \end{equation}

    Now that these relations are inverted. To do this, the following relations are also needed:

    \begin{equation}
        \bar{u}_{r} (p`') u_{s} = \delta_{rs}
    \end{equation}

    \begin{equation}
        \bar{v}_{r} v_{s} = - \delta_{rs}
    \end{equation}

    \begin{equation}
        \bar{v}_{r} u_{s} = 0
    \end{equation}

    \begin{equation}
        \bar{u}_{r} v_{s} = 0
    \end{equation}

    Also, we need these identities:

    \begin{equation}
        \overline{u}_{r} (\vec{p}) \gamma^{0} \overline{u}_{s} (\vec{p}) = \overline{u}_{r}^{\dagger} (\vec{p}) \overline{u}_{s} (\vec{p}) = \frac{\omega}{m} \delta_{rs}
    \end{equation}

    \begin{equation}
        \overline{v}_{r} (\vec{p}) \gamma^{0} \overline{v}_{s} (\vec{p}) = \overline{v}_{r}^{\dagger} (\vec{p}) \overline{v}_{s} (\vec{p}) = \frac{\omega}{m} \delta_{rs}
    \end{equation}

    Which can be proved as follows:

    \begin{equation}
        \overline{u}_{r} (\vec{p}) \gamma^{0} \overline{u}_{s} (\vec{p}) = \overline{u}_{r}^{\dagger} (\vec{p}) \overline{u}_{s} (\vec{p})
    \end{equation}

    \begin{equation}
        \overline{u}_{r}^{\dagger} (\vec{p}) \overline{u}_{s} (\vec{p}) = \overline{u}_{r} (\vec{p}) \gamma^{0} \overline{u}_{s} (\vec{p}) 
    \end{equation}

    \begin{equation}
        \begin{aligned} []
            (\gamma^\mu p_\mu - m) u_s (p) & \rightarrow & \overline{u}_s (p) (\gamma^\mu p_\mu - m) \\
            \downarrow & & \downarrow \\
            u_s (p) = \frac{1}{m} \gamma^\mu p_\mu u_s (p) & \rightarrow & \overline{u}_s (p) = \frac{1}{m} \overline{u}_s (p) \gamma^\mu p_\mu \\
        \end{aligned}
    \end{equation}

    \begin{equation}
        \begin{aligned} []
            \overline{u}_{r}^{\dagger} (\vec{p}) \overline{u}_{s} (\vec{p}) & = \frac{1}{2} ( \overline{u}_{r} (\vec{p}) \gamma^{0} \overline{u}_{s} (\vec{p}) + \overline{u}_{r} (\vec{p}) \gamma^{0} \overline{u}_{s} (\vec{p}) ) \\
            & = \frac{1}{2} ( \overline{u}_{r} (\vec{p}) \gamma^{0} \gamma^\mu p_\mu \overline{u}_{s} (\vec{p}) + \overline{u}_{r} (\vec{p}) \gamma^{0} \gamma^\mu p_\mu \overline{u}_{s} (\vec{p}) ) \\
            & = \overline{u}_{r} (\vec{p}) \{ \gamma^0, \gamma^\mu \} \overline{u}_{s} (\vec{p})
        \end{aligned}
    \end{equation}

    \begin{equation}
        \{ \gamma^0, \gamma^\mu \} = 2 \eta^{0 \mu}
    \end{equation}

    \begin{equation}
        u^{\dagger}_r (p) u_s (p) = \frac{\omega}{m} \overline{u}_r (p) u_s (p) = \frac{\omega}{m} \delta_{r s}
    \end{equation}

    Let's Start by finding the formula for $a_{r} (p)$:

    \begin{equation}
        \psi (x, t) = \sum_{s} \int \frac{m}{\omega} [a_s (\vec{p}) u_s (\vec{p}) e^{- i p \cdot x} + b^{\dagger}_s (\vec{p}) v_s (\vec{p}) e^{i p \cdot x}] \frac{d^{3} p}{(2 \pi)^3}
    \end{equation}

    \begin{equation}
        \int \psi (x, t) e^{i p \cdot x} d^3 x = \sum_{s} \int \frac{m}{\omega} [a_s (\vec{p}) u_s (\vec{p}) e^{- i (\omega - \omega') \cdot x} + b^{\dagger}_s (\vec{p}) v_s (\vec{p}) e^{i (\omega - \omega') \cdot x}] \frac{d^{3} p}{(2 \pi)^3} d^3 x
    \end{equation}

    \begin{equation}
        \int \psi (x, t) e^{i p \cdot x} d^3 x = \sum_{s} \int \frac{m}{\omega} [a_s (\vec{p}) u_s (\vec{p}) e^{- i (\omega - \omega') \cdot x} e^{i (p - p') \cdot x} + b^{\dagger}_s (\vec{p}) v_s (\vec{p}) e^{i (\omega - \omega') \cdot x} e^{- i (p - p') \cdot x}] \frac{d^{3} p}{(2 \pi)^3} d^3 x
    \end{equation}

    The X integration can be done with the following formulas:

    \begin{equation}
        \int e^{- i (\vec{k} - \vec{k'} \cdot \vec{x})} d^3 x = \delta^3 (\vec{k} - \vec{k'}) \int e^{- i (\vec{k} - \vec{k'} \cdot \vec{x})} \frac{d^3 x}{(2 \pi)^3} = \delta^3 (\vec{k} - \vec{k'})
    \end{equation}

    Applying them gives:

    \begin{equation}
        \int \psi (x, t) e^{i p \cdot x} d^3 x = \sum_{s} \int \frac{m}{\omega} [a_s (\vec{p}) u_s (\vec{p}) e^{- i (\omega - \omega') \cdot x} \delta^3 (\vec{p} - \vec{p'}) + b^{\dagger}_s (\vec{p}) v_s (\vec{p}) e^{i (\omega - \omega') \cdot x} \delta^3 (\vec{p} - \vec{p'})] \frac{d^{3} p}{(2 \pi)^3} d^3 x
    \end{equation}

    Now the P integration can be done:

    \begin{equation}
        \int \psi (x, t) e^{i p \cdot x} d^3 x = \sum_{s} \int \frac{m}{\omega} [a_s (\vec{p}) u_s (\vec{p}) + b^{\dagger}_s (\vec{p}) v_s (\vec{p}) e ^{2 i \omega' t}] \frac{d^{3} p}{(2 \pi)^3} d^3 x
    \end{equation}

    \begin{equation}
        \int \psi (x, t) e^{i p \cdot x} d^3 x = \sum_{s} \int \frac{m}{\omega} [a_s (- \vec{p}) u_s (\vec{p}) + b^{\dagger}_s (- \vec{p}) v_s (\vec{p}) e ^{2 i \omega' t}] \frac{d^{3} p}{(2 \pi)^3} d^3 x
    \end{equation}

    \begin{equation}
        \int \overline{u}_{r} (\vec{p}) \psi (x, t) e^{i p \cdot x} d^3 x = \sum_{s} \int \frac{m}{\omega} [a_s (- \vec{p}) \overline{u}_{r} (\vec{p}) u_s (\vec{p}) + b^{\dagger}_s (- \vec{p}) \overline{u}_{r} (\vec{p}) v_s (\vec{p}) e ^{2 i \omega' t}] \frac{d^{3} p}{(2 \pi)^3} d^3 x
    \end{equation}

    \begin{equation}
        \overline{u}_{r} (\vec{p}) v_{s} (\vec{p}) = 0
    \end{equation}

    \begin{equation}
        \int \overline{u}_{r} (\vec{p}) \psi (x, t) e^{i p \cdot x} d^3 x = \sum_{s} \int \frac{m}{\omega} a_s (- \vec{p}) \overline{u}_{r} (\vec{p}) u_s (\vec{p}) \frac{d^{3} p}{(2 \pi)^3} d^3 x
    \end{equation}

    \begin{equation}
        \begin{aligned}[]
            \overline{u}_{r} (\vec{p}) = \overline{u}_{r} (- \vec{p}) \gamma^0 & & \overline{u}_{s} (- \vec{p}) = \gamma^0 \overline{u}_{s} (\vec{p})
        \end{aligned}
    \end{equation}

    \begin{equation}
        \int \overline{u}_{r} (\vec{p}) \gamma^0 \psi (x, t) e^{i p \cdot x} d^3 x = \frac{m}{\omega} a_s (- \vec{p}) \overline{u}_{r} (\vec{p}) \gamma^0 \overline{u}_{s} (\vec{p})
    \end{equation}

    \begin{equation}
        \overline{u}_{r} (\vec{p}) \gamma^0 \overline{u}_{s} (\vec{p}) = \frac{m}{\omega} \delta_{rs}
    \end{equation}

    \begin{equation}
        \int \overline{u}_{r} (\vec{p}) \gamma^0 \psi (x, t) e^{i p \cdot x} d^3 x = \sum_s a_s (- \vec{p}) \delta_{rs} a_r (- \vec{p})
    \end{equation}

    \begin{framed}
        \begin{equation}
            a_r (\vec{p}) = \int \overline{u}_{r} (\vec{p}) \gamma^0 \psi (x, t) e^{i p \cdot x} d^3 x
        \end{equation}
    \end{framed}

    Taking the hermitian conjugate gives the result for $a_{r}^{\dagger} (\vec{p})$:

    \begin{equation}
        a_r (\vec{p}) = \int u^{\dagger} (\vec{p}) \gamma^0 \psi (x, t) e^{i p \cdot x} d^3 x
    \end{equation}

    \begin{equation}
        \int \psi (x, t) \gamma^0 u^{\dagger} (\vec{p}) e^{i p \cdot x} d^3 x = a_r^\dagger (\vec{p})
    \end{equation}
    
    \begin{framed}
        \begin{equation}
            \overline{a}_r (\vec{p}) = \int \psi (x, t) \gamma^0 u^{\dagger} (\vec{p}) e^{i p \cdot x} d^3 x
        \end{equation}
    \end{framed}

    Now let's solve or $b_{r} (\vec{p})$:

    \begin{equation}
        \psi (x, t) = \sum_{s} \int \frac{m}{\omega} [b_s (\vec{p}) v_s (\vec{p}) e^{- i p \cdot x} + a^{\dagger}_s (\vec{p}) u_s (\vec{p}) e^{i p \cdot x}] \frac{d^{3} p}{(2 \pi)^3}
    \end{equation}

    \begin{equation}
        \int \overline{\psi} (x, t) e^{i p \cdot x} d^3 x = \sum_{s} \int \frac{m}{\omega} [b_s (\vec{p}) v_s (\vec{p}) e^{- i (p - p')} + a^{\dagger}_s (\vec{p}) u_s (\vec{p}) e^{i (p - p')}] \frac{d^{3} p}{(2 \pi)^3} d^3 x
    \end{equation}

    \begin{equation}
        \int \overline{\psi} (x, t) e^{i p \cdot x} d^3 x = \sum_{s} \int \frac{m}{\omega} [b_s (\vec{p}) v_s (\vec{p}) e^{- i (\omega - \omega') \cdot t} e^{i (p - p') \cdot x} + a^{\dagger}_s (\vec{p}) u_s (\vec{p}) e^{i (\omega - \omega') \cdot t} e^{- i (p - p') \cdot x}] \frac{d^{3} p}{(2 \pi)^3} d^3 x
    \end{equation}

    The $x$ integration can be done with the following formulas:

    \begin{equation}
        \int e^{-i (k - k') \cdot x} \frac{d^3 x}{(2 \pi)^3} = \delta^{3} (\vec{k} - \vec{k'})
    \end{equation}

    \begin{equation}
        \int e^{-i (k + k') \cdot x} \frac{d^3 x}{(2 \pi)^3} = \delta^{3} (\vec{k} + \vec{k'})
    \end{equation}

    This gives:

    \begin{equation}
        \int \overline{\psi} (x, t) e^{i p \cdot x} d^3 x = \sum_{s} \int \frac{m}{\omega} [b_s (\vec{p}) v_s (\vec{p}) e^{- i (\omega - \omega') \cdot t} \delta^{3} (\vec{p} - \vec{p'}) + a^{\dagger}_s (\vec{p}) u_s (\vec{p}) e^{i (\omega - \omega') \cdot t} \delta^{3} (\vec{p} + \vec{p'})] \frac{d^{3} p}{(2 \pi)^3} d^3 x
    \end{equation}

    Now for the $p$ integration:

    \begin{equation}
        \int \overline{\psi} (x, t) e^{i p \cdot x} d^3 x = \sum_{s} \int \frac{m}{\omega} [b_s (\vec{p}) v_s (\vec{p}) e^{- i (\omega - \omega') \cdot t} + a^{\dagger}_s (\vec{p}) u_s (\vec{p}) e^{i (\omega - \omega') \cdot t}] \frac{d^{3} p}{(2 \pi)^3} d^3 x
    \end{equation}

    \begin{equation}
        \int \overline{\psi} (x, t) e^{i p \cdot x} d^3 x = \sum_{s} \int \frac{m}{\omega} [b_s (- \vec{p}) v_s (- \vec{p}) e^{- i (\omega - \omega') \cdot t} + a^{\dagger}_s (\vec{p}) u_s (\vec{p}) e^{i (\omega - \omega') \cdot t}] \frac{d^{3} p}{(2 \pi)^3} d^3 x
    \end{equation}

    \begin{equation}
        \int \overline{\psi} (x, t) v_{s} (\vec{p}) e^{i p \cdot x} d^3 x = \sum_{s} \int \frac{m}{\omega} [b_s (- \vec{p}) v_s (- \vec{p}) e^{- i (\omega - \omega') \cdot t}  v_{s} (\vec{p}) + a^{\dagger}_s (\vec{p}) u_s (\vec{p}) v_{s} (\vec{p}) e^{i (\omega - \omega') \cdot t}] \frac{d^{3} p}{(2 \pi)^3} d^3 x
    \end{equation}

    \begin{equation}
        \overline{u}_{r} (\vec{p}) v_{s} (\vec{p}) = 0
    \end{equation}

    \begin{equation}
        \int \overline{\psi} (x, t) v_{s} (\vec{p}) e^{i p \cdot x} d^3 x = \sum_{s} \int \frac{m}{\omega} b_s (- \vec{p}) v_s (- \vec{p}) e^{- i (\omega - \omega') \cdot t}  v_{s} (\vec{p}) \frac{d^{3} p}{(2 \pi)^3} d^3 x
    \end{equation}

    \begin{equation}
        \begin{aligned}[]
            v_{r} (\vec{p}) = \gamma^0 \overline{v}_{r} (- \vec{p}) & & v_{s} (- \vec{p}) = \overline{v}_{s} (\vec{p}) \gamma^0
        \end{aligned}
    \end{equation}

    \begin{equation}
        - \int \overline{\psi} (x, t) v_{s} (\vec{p}) e^{i p \cdot x} d^3 x = - \sum_{s} \int \frac{m}{\omega} b_s (- \vec{p}) v_s (- \vec{p}) e^{- i (\omega - \omega') \cdot t}  v_{s} (\vec{p}) \frac{d^{3} p}{(2 \pi)^3} d^3 x
    \end{equation}

    \begin{equation}
        \begin{aligned}[]
            v_{r} (\vec{p}) = - \gamma^0 \overline{v}_{r} (- \vec{p}) & & v_{s} (- \vec{p}) = - \overline{v}_{s} (\vec{p}) \gamma^0
        \end{aligned}
    \end{equation}

    \begin{equation}
        \overline{v}_{r} (\vec{p}) \gamma^0 \overline{v}_{s} (\vec{p}) = \frac{m}{\omega} \delta_{rs}
    \end{equation}

    \begin{equation}
        \int \psi (x, t) \gamma^0 v (\vec{p}) e^{i p \cdot x} d^3 x = b_r (\vec{p})
    \end{equation}
    
    \begin{framed}
        \begin{equation}
            b_r (\vec{p}) = \int \psi (x, t) \gamma^0 v (\vec{p}) e^{i p \cdot x} d^3 x
        \end{equation}
    \end{framed}

    Taking the hermitian conjugate gives the result for $b_{r}^{\dagger} (\vec{p})$:

    \begin{equation}
        b_r (\vec{p}) = \int \psi (x, t) \gamma^0 v (\vec{p}) e^{i p \cdot x} d^3 x
    \end{equation}

    \begin{equation}
        b_r^\dagger (\vec{p}) = \int v^{\dagger} (\vec{p}) \gamma^0 \psi (x, t) e^{i p \cdot x} d^3 x
    \end{equation}

    \begin{framed}
        \begin{equation}
            \overline{b_r} (\vec{p}) = \int v^{\dagger} (\vec{p}) \gamma^0 \psi (x, t) e^{i p \cdot x} d^3 x
        \end{equation}
    \end{framed}

    So, the complete set is as follows:

    \begin{framed}
        \begin{equation}
            \begin{aligned}[]
                a_r (\vec{p}) = \int \overline{u}_{r} (\vec{p}) \gamma^0 \psi (x, t) e^{i p \cdot x} d^3 x & & b_r (\vec{p}) = \int \psi (x, t) \gamma^0 v (\vec{p}) e^{i p \cdot x} d^3 x \\
                \overline{a}_r (\vec{p}) = \int \psi (x, t) \gamma^0 u^{\dagger} (\vec{p}) e^{i p \cdot x} d^3 x & & \overline{b_r} (\vec{p}) = \int v^{\dagger} (\vec{p}) \gamma^0 \psi (x, t) e^{i p \cdot x} d^3 x
            \end{aligned}
        \end{equation}
    \end{framed}

    Quantizing with equal time commutation relation gives unphysical answers. To get results that make sense, one must quantize with equal time anticommutation relations. This turns out to be
    necessary any time a half integer spin is being quantized. It has the effect of causing the field quanta to obey Fermi-Dirac statistics, and therefore be Fermions, as we shall see. 
    Additionally, quantizing integer spin fields with equal time anticommutation relation gives similar unphysical results. The effect is called the spin-statistics theorem. Replacing
    commutators with anticommutators gives the following anticommutation relations:

    \begin{equation}
        \{ \psi_i (\vec{x}, t), \pi_{\psi_j} (\vec{x}, t) \} = i \delta^{3} (\bar{x} - \bar{x'}) \delta_{ij}
    \end{equation}

    \begin{equation}
        \{ \psi_i (\vec{x}, t), \psi_j^{\dagger} (\vec{x}, t) \} = \delta^{3} (\bar{x} - \bar{x'}) \delta_{ij}
    \end{equation}

    \begin{equation}
        \{ \pi_{\psi_i} (\vec{x}, t), \pi_{\psi_j} (\vec{x}, t) \} = 0
    \end{equation}

    \begin{equation}
        \{ \psi_i (\vec{x}, t), \psi_j (\vec{x}, t) \} = 0
    \end{equation}

    Now, we can start evaluating the Fourier Coefficient anticommutators. Let's evaluate $\{a_r (\vec{p}), a_s^{\dagger} (\vec{p'})\}$ first:

    \begin{equation}
        \{a_r (\vec{p}), a_s^{\dagger} (\vec{p'})\} = \int e^{i p \cdot x - i p' \cdot x'} [u_r (\vec{p}) \gamma^0 \psi (\vec{x}, t) \overline{\psi} (\vec{x}, t) \gamma^0 u_s (\vec{p'}) + \overline{\psi} (\vec{x}, t) \gamma^0 u_s (\vec{p'}) \overline{u}_r (\vec{p'}) \psi (\vec{x}, t)] \; d^3 x d^3 x'
    \end{equation}

    \begin{equation}
        \{a_r (\vec{p}), a_s^{\dagger} (\vec{p'})\} = \int e^{i p \cdot x - i p' \cdot x'} [u_r (\vec{p})  \psi (\vec{x}, t) \psi^{\dagger} (\vec{x}, t)  u_s (\vec{p'}) + \psi^{\dagger} (\vec{x}, t)  u_s (\vec{p'}) u^{\dagger}_r (\vec{p'}) \psi (\vec{x}, t)] \; d^3 x d^3 x'
    \end{equation}

    Where $t = t'$. Manipulating, and un-supressing the spinor index gives:

    \begin{equation}
        \{a_r (\vec{p}), a_s^{\dagger} (\vec{p'})\} = \int e^{i p \cdot x - i p' \cdot x'} [u_{ri} (\vec{p})  \psi_i (\vec{x}, t) \psi^{\dagger}_j (\vec{x}, t)  u_{sj} (\vec{p'}) + \psi^{\dagger}_j (\vec{x}, t)  u_{sj} (\vec{p'}) u^{\dagger}_{ri} (\vec{p'}) \psi_i (\vec{x}, t)] \; d^3 x d^3 x'
    \end{equation}

    \begin{equation}
        \{a_r (\vec{p}), a_s^{\dagger} (\vec{p'})\} = \int e^{i p \cdot x - i p' \cdot x'} [\psi_i (\vec{x}, t) \psi^{\dagger}_j (\vec{x}, t) + \psi^{\dagger}_j (\vec{x}, t) \psi_i (\vec{x}, t)] \; d^3 x d^3 x'
    \end{equation}

    \begin{equation}
        \{a_r (\vec{p}), a_s^{\dagger} (\vec{p'})\} = \int e^{i p \cdot x - i p' \cdot x'} \{ \psi_i (\vec{x}, t), \psi^{\dagger}_j (\vec{x}, t) \} \; d^3 x d^3 x'
    \end{equation}

    Now applying equal time anticommutation relations:

    \begin{equation}
        \{a_r (\vec{p}), a_s^{\dagger} (\vec{p'})\} = \int e^{i p \cdot x - i p' \cdot x'} u_{ri} (\vec{p}) \delta^{3} (\vec{x} - \vec{x'}) \delta_{ij} u_{si} (\vec{p'}) \; d^3 x d^3 x'
    \end{equation}

    \begin{equation}
        \{a_r (\vec{p}), a_s^{\dagger} (\vec{p'})\} = \int e^{i p \cdot x - i p' \cdot x'} u_{ri} (\vec{p}) u_{si} (\vec{p'}) \delta^{3} (\vec{x} - \vec{x'}) \; d^3 x d^3 x'
    \end{equation}

    Now, supressing the spinor indices:

    \begin{equation}
        \{a_r (\vec{p}), a_s^{\dagger} (\vec{p'})\} = \int e^{i p \cdot x - i p' \cdot x'} u_r (\vec{p}) u_s (\vec{p'}) \delta^{3} (\vec{x} - \vec{x'}) \; d^3 x d^3 x'
    \end{equation}

    Doing the $\vec{x}$ integration gives:

    \begin{equation}
        \{a_r (\vec{p}), a_s^{\dagger} (\vec{p'})\} = \int e^{i (p - p') \cdot x'} u_r (\vec{p}) u_s (\vec{p'}) \; d^3 x d^3 x'
    \end{equation}

    Now, we can use the following integral to do the x-integration:

    \begin{equation}
        \int e^{- i (\vec{k} - \vec{k'} \cdot \vec{x})} d^3 x = \delta^3 (\vec{k} - \vec{k'}) \int e^{- i (\vec{k} - \vec{k'} \cdot \vec{x})} \frac{d^3 x}{(2 \pi)^3} = (2 \pi)^3 \delta^3 (\vec{k} - \vec{k'})
    \end{equation}

    Applying this gives:

    \begin{equation}
        \{a_r (\vec{p}), a_s^{\dagger} (\vec{p'})\} = (2 \pi)^3 \delta^3 (\vec{k} - \vec{k'}) u_r (\vec{p}) u_s (\vec{p'}) \; d^3 x d^3 x'
    \end{equation}

    Then we also have $u_r^{\dagger} (\vec{p}) u_s (\vec{p}) = \frac{\omega}{m} \delta_{rs}$:

    \begin{framed}
        \begin{equation}
            \{a_r (\vec{p}), a_s^{\dagger} (\vec{p'})\} = (2 \pi)^3 \delta^3 (\vec{k} - \vec{k'}) \frac{\omega}{m} \delta_{rs} \; d^3 x d^3 x'
        \end{equation}
    \end{framed}

    All other Fourier coefficient anticommutators trivially evaluate to zero, given that

    \begin{equation}
        \{ \pi_{\psi_i} (\vec{x}, t), \pi_{\psi_j} (\vec{x'}, t) \} = 0
    \end{equation} 

    and

    \begin{equation}
        \{ \psi_i (\vec{x}, t), \psi_j (\vec{x}, t) \} = 0
    \end{equation}

    So, the complete set of Fourier coefficient anticommutation relations is as follows:

    \begin{equation}
        \{a_r (\vec{p}), a_s^{\dagger} (\vec{p'})\} = (2 \pi)^3 \delta^3 (\vec{k} - \vec{k'}) \frac{\omega}{m} \delta_{rs} \; d^3 x d^3 x'
    \end{equation}

    \begin{equation}
        \{b_r (\vec{p}), b_s^{\dagger} (\vec{p'})\} = (2 \pi)^3 \delta^3 (\vec{k} - \vec{k'}) \frac{\omega}{m} \delta_{rs} \; d^3 x d^3 x'
    \end{equation}

    \begin{equation}
        \{a_r (\vec{p}), a_s (\vec{p'})\} = 0 \qquad \{a_r^{\dagger} (\vec{p}), a_s^{\dagger} (\vec{p'})\} = 0
    \end{equation}

    \begin{equation}
        \{a_r (\vec{p}), b_s (\vec{p'})\} = 0 \qquad \{a_r^{\dagger} (\vec{p}), b_s^{\dagger} (\vec{p'})\} = 0
    \end{equation}

    \begin{equation}
        \{a_r^{\dagger} (\vec{p}), b_s (\vec{p'})\} = 0 \qquad \{a_r (\vec{p}), b_s^{\dagger} (\vec{p'})\} = 0
    \end{equation}

    With these anticommutation relations established, the states of the theory can be worked out. The first step is to express
    the Hamiltonian in terms of the Fourier coefficients. The process of doing this starts with the expression for the Hamiltonian
    given above:

    \begin{equation}
        H = \int i \psi^{\dagger} \partial_{0} \psi d^3 x
    \end{equation}

    Now we can insert the values of the fields:

    \begin{equation}
        \partial_{0} \psi (x, t) = \sum_{s} \int \frac{m}{\omega} [a_s (\vec{p}) u_s (\vec{p}) e^{- i p \cdot x} + b_s^{\dagger} (\vec{p}) v_s (\vec{p}) e^{i p \cdot x}] \frac{d^{3} p}{(2 \pi)^3}
    \end{equation}

    \begin{equation}
        \psi^{\dagger} (x, t) = \sum_{s} \int \frac{m}{\omega} [b_s (\vec{p}) v_s^{\dagger} (\vec{p}) e^{- i p \cdot x} + a_s^{\dagger} (\vec{p}) u_s^{\dagger} (\vec{p}) e^{i p \cdot x}] \frac{d^{3} p}{(2 \pi)^3}
    \end{equation}

    Inserting these gives:

    \begin{equation}
        H = i \sum_{s} \int \frac{m}{\omega} [a_s (\vec{p}) u_s (\vec{p}) e^{- i p \cdot x} + b_s^{\dagger} (\vec{p}) v_s (\vec{p}) e^{i p \cdot x}] [b_s (\vec{p}) v_s^{\dagger} (\vec{p}) e^{- i p \cdot x} + a_s^{\dagger} (\vec{p}) u_s^{\dagger} (\vec{p}) e^{i p \cdot x}] \frac{d^{3} p}{(2 \pi)^3} d^3 x
    \end{equation}


    \begin{equation}
        \begin{aligned}
            H = i \sum_{s} \int \frac{m}{\omega} \\
        [b_r (\vec{p}) a_s (\vec{p}) v_r^{\dagger} (\vec{p}) u_s^{\dagger} (\vec{p}) e^{- i p \cdot x} \\
        + b_r (\vec{p}) b_s^{\dagger} (\vec{p}) v_r^{\dagger} (\vec{p}) v_s (\vec{p}) e^{i p \cdot x} \\
        + a_r^{\dagger} (\vec{p}) a_s (\vec{p}) u_r^{\dagger} (\vec{p}) u_s (\vec{p}) e^{- i p \cdot x} \\
        a_r^{\dagger} (\vec{p}) b_s^{\dagger} (\vec{p}) u_r^{\dagger} (\vec{p}) v_s (\vec{p}) e^{i p \cdot x}] \\
        \frac{d^{3} p}{(2 \pi)^3} d^3 x
        \end{aligned}
    \end{equation}

    Now, the following relations can be used to do the integration over x:

    \begin{equation}
        \int e^{-i (k-k') \cdot x} d^3 x = \int e^{-i (\omega-\omega') \cdot x} e^{-i (k-k') \cdot x} d^3 x = e^{-i (\omega-\omega') \cdot x} (2 \pi^3) \delta^3 (k - k') = (2 \pi^3) \delta^3 (k - k')
    \end{equation}

    \begin{equation}
        \int e^{-i (k-k') \cdot x} d^3 x = \int e^{-i (\omega-\omega') \cdot x} e^{-i (k-k') \cdot x} d^3 x = e^{-i (\omega-\omega') \cdot x} (2 \pi^3) \delta^3 (k + k') = e^{-2i \omega t} (2 \pi^3) \delta^3 (k + k')
    \end{equation}

    Applying these:

    \begin{equation}
        \begin{aligned}
            H = i \sum_{s} \int \frac{m}{\omega} \\
        [b_r (\vec{p}) a_s (\vec{p}) v_r^{\dagger} (\vec{p}) u_s^{\dagger} (\vec{p}) e^{-2i \omega t} (2 \pi^3) \delta^3 (p + p') \\
        + b_r (\vec{p}) b_s^{\dagger} (\vec{p}) v_r^{\dagger} (\vec{p}) v_s (\vec{p}) \delta^3 (p - p') \\
        + a_r^{\dagger} (\vec{p}) a_s (\vec{p}) u_r^{\dagger} (\vec{p}) u_s (\vec{p}) \delta^3 (p - p') \\
        a_r^{\dagger} (\vec{p}) b_s^{\dagger} (\vec{p}) u_r^{\dagger} (\vec{p}) v_s (\vec{p}) e^{-2i \omega t} (2 \pi^3) \delta^3 (p + p')] \\
        \frac{d^{3} p}{(2 \pi)^3} d^3 x
        \end{aligned}
    \end{equation}

    Now the p' integration can be done:

    \begin{equation}
        \begin{aligned}
            H = i \sum_{s} \int \frac{m}{\omega} \\
        [b_r (\vec{p}) a_s (\vec{p}) v_r^{\dagger} (\vec{p}) u_s^{\dagger} e^{-2i \omega t} (2 \pi^3) (\vec{p}) \\
        + b_r (\vec{p}) b_s^{\dagger} (\vec{p}) v_r^{\dagger} (\vec{p}) v_s (\vec{p}) \\
        + a_r^{\dagger} (\vec{p}) a_s (\vec{p}) u_r^{\dagger} (\vec{p}) u_s (\vec{p}) \\
        a_r^{\dagger} (\vec{p}) b_s^{\dagger} (\vec{p}) u_r^{\dagger} (\vec{p}) v_s (\vec{p}) e^{-2i \omega t} (2 \pi^3)] \\
        \frac{d^{3} p}{(2 \pi)^3} d^3 x
        \end{aligned}
    \end{equation}

    \begin{equation}
        - \gamma^0 v_r (\vec{p}) = v_r (- \vec{p}) \qquad u_s (- \vec{p}) = u_s (\vec{p})
    \end{equation}

    \begin{equation}
        \begin{aligned}
            H = i \sum_{s} \int \frac{m}{\omega} \\
        [b_r (\vec{p}) a_s (\vec{p}) v_r^{\dagger} (\vec{p}) u_s^{\dagger} e^{-2i \omega t} (2 \pi^3) (\vec{p}) \\
        + b_r (\vec{p}) b_s^{\dagger} (\vec{p}) v_r^{\dagger} (\vec{p}) v_s (\vec{p}) \\
        + a_r^{\dagger} (\vec{p}) a_s (\vec{p}) u_r^{\dagger} (\vec{p}) u_s (\vec{p}) \\
        a_r^{\dagger} (\vec{p}) b_s^{\dagger} (\vec{p}) u_r^{\dagger} (\vec{p}) v_s (\vec{p}) e^{-2i \omega t} (2 \pi^3)] \\
        \frac{d^{3} p}{(2 \pi)^3} d^3 x
        \end{aligned}
    \end{equation}

    \begin{equation}
        \overline{v}_r (\vec{p}) u_s (\vec{p}) = 0 \qquad \overline{u}_r (\vec{p}) v_s (\vec{p}) = 0
    \end{equation}

    \begin{equation}
        \begin{aligned}
            H = i \sum_{s} \int \frac{m}{\omega} [b_r (\vec{p}) b_s^{\dagger} (\vec{p}) v_r^{\dagger} (\vec{p}) v_s (\vec{p}) \\
        + a_r^{\dagger} (\vec{p}) a_s (\vec{p}) u_r^{\dagger} (\vec{p}) u_s (\vec{p})] \frac{d^{3} p}{(2 \pi)^3} d^3 x
        \end{aligned}
    \end{equation}

    \begin{equation}
        u_r^{\dagger} (\vec{p}) u_s (\vec{p}) = \frac{\omega}{m} \delta_{rs} \qquad v_r^{\dagger} (\vec{p}) v_s (\vec{p}) = \frac{\omega}{m} \delta_{rs}
    \end{equation}

    \begin{equation}
        \begin{aligned}
            H = i \sum_{s} \int \frac{m}{\omega} [b_r (\vec{p}) b_s^{\dagger} (\vec{p}) \frac{\omega}{m} \delta_{rs} \\
        + a_r^{\dagger} (\vec{p}) a_s (\vec{p}) \frac{\omega}{m} \delta_{rs}] \frac{d^{3} p}{(2 \pi)^3} d^3 x
        \end{aligned}
    \end{equation}

    \begin{equation}
        \begin{aligned}
            H = i \sum_{s} \int \frac{m}{\omega} [b_r (\vec{p}) b_s^{\dagger} (\vec{p}) \\
        + a_r^{\dagger} (\vec{p}) a_s (\vec{p})] \frac{d^{3} p}{(2 \pi)^3} d^3 x
        \end{aligned}
    \end{equation}

    \begin{equation}
        \begin{aligned}
            H = i \sum_{s} \int \frac{m}{\omega} [b_s^{\dagger} (\vec{p}) b_r (\vec{p}) \\
        + a_r^{\dagger} (\vec{p}) a_s (\vec{p}) - \{ b_s (\vec{p}), b_s^{\dagger} (\vec{p}) \}] \frac{d^{3} p}{(2 \pi)^3} d^3 x
        \end{aligned}
    \end{equation}

    This can be re-expressed using $\{ b_s (\vec{p}), b_s^{\dagger} (\vec{p}) \} = (2 \pi)^3 \frac{\omega}{m} \delta^3 (\vec{p} - \vec{p'}) \delta_{rs}$ as follows:

    \begin{equation}
        \begin{aligned}
            H = i \sum_{s} \int \frac{m}{\omega} [b_s^{\dagger} (\vec{p}) b_r (\vec{p}) \\
        + a_r^{\dagger} (\vec{p}) a_s (\vec{p}) - (2 \pi)^3 \frac{\omega}{m} \delta^3 (\vec{p} - \vec{p'}) \delta_{rs}] \frac{d^{3} p}{(2 \pi)^3} d^3 x
        \end{aligned}
    \end{equation}

    Infinite (and finite) shifts in the zero point of the energy scale cannot be measured, so the last term can be ignored. Doing this gives the normal ordering Hamiltonian:

    \begin{framed}
        \begin{equation}
            \begin{aligned}
                H = i \sum_{s} \int \frac{m}{\omega} [b_s^{\dagger} (\vec{p}) b_r (\vec{p}) \\
            + a_r^{\dagger} (\vec{p}) a_s (\vec{p}) - (2 \pi)^3 \frac{\omega}{m} \delta^3 (\vec{p} - \vec{p'}) \delta_{rs}] \frac{d^{3} p}{(2 \pi)^3} d^3 x
            \end{aligned}
        \end{equation}
    \end{framed}

    In the case of spin half fields, normal ordering is formally defined as, at the cost of a minus sign every time operators are moved past each other, positioning all
    $a_s^{\dagger} (\vec{p})$ and $b_s^{\dagger} (\vec{p})$ operators (later we weill see that these are creation operators) to the left and all $a_s (\vec{p})$ and
    $b_s (\vec{p})$ (these will turn out to be annihilation operators to the right). Doing this yields more sensible quantum physical expressions. The big difference
    between normal ordering bosonic operators and fermionic oerpators is the minus sign that shows up under interchange of fermionic operators.

    While we are at it, it is worth expressing the momentum and charge introduced above in terms of the Fourier coefficients. Let's start with the momentum. above, we found
    that the momentum is equal to:

    \begin{equation}
        \vec{P} = - i \int \psi^{\dagger} \nabla \psi d^3 x
    \end{equation}

    Now we can insert the values of the fields:

    \begin{equation}
        \nabla \psi (x, t) = \sum_{s} \int \frac{m}{\omega} [a_s (\vec{p}) u_s (\vec{p}) e^{- i p \cdot x} - b_s^{\dagger} (\vec{p}) v_s (\vec{p}) e^{i p \cdot x}] \frac{d^{3} p}{(2 \pi)^3}
    \end{equation}

    \begin{equation}
        \psi^{\dagger} (x, t) = \sum_{s} \int \frac{m}{\omega} [b_s (\vec{p}) v_s^{\dagger} (\vec{p}) e^{- i p \cdot x} + a_s^{\dagger} (\vec{p}) u_s^{\dagger} (\vec{p}) e^{i p \cdot x}] \frac{d^{3} p}{(2 \pi)^3}
    \end{equation}

    Inserting these:

    \begin{equation}
        \begin{aligned}
            \vec{P} = - i \int \sum_{s} \int \frac{m}{\omega} [b_r (\vec{p}) a_s (\vec{p}) v_r^{\dagger} (\vec{p}) u_s^{\dagger} (\vec{p}) e^{- i p \cdot x} \\
            + b_r (\vec{p}) b_s^{\dagger} (\vec{p}) v_r^{\dagger} (\vec{p}) v_s (\vec{p}) e^{i p \cdot x} \\
            + a_r^{\dagger} (\vec{p}) a_s (\vec{p}) u_r^{\dagger} (\vec{p}) u_s (\vec{p}) e^{- i p \cdot x} \\
            a_r^{\dagger} (\vec{p}) b_s^{\dagger} (\vec{p}) u_r^{\dagger} (\vec{p}) v_s (\vec{p}) e^{i p \cdot x}] \frac{d^{3} p}{(2 \pi)^3} \frac{d^{3} p'}{(2 \pi)^3} d^3 x
        \end{aligned}
    \end{equation}

    Now, the following relations can be used to do the integration over x:

    \begin{equation}
        \int e^{-i (k-k') \cdot x} d^3 x = \int e^{-i (\omega-\omega') \cdot x} e^{-i (k-k') \cdot x} d^3 x = e^{-i (\omega-\omega') \cdot x} (2 \pi^3) \delta^3 (k - k') = (2 \pi^3) \delta^3 (k - k')
    \end{equation}

    \begin{equation}
        \int e^{-i (k-k') \cdot x} d^3 x = \int e^{-i (\omega-\omega') \cdot x} e^{-i (k-k') \cdot x} d^3 x = e^{-i (\omega-\omega') \cdot x} (2 \pi^3) \delta^3 (k + k') = e^{-2i \omega t} (2 \pi^3) \delta^3 (k + k')
    \end{equation}

    Applying these:

    \begin{equation}
        \begin{aligned}
            \vec{P} = - i \int \sum_{s} \int \frac{m}{\omega} [b_r (\vec{p}) a_s (\vec{p}) v_r^{\dagger} (\vec{p}) u_s^{\dagger} (\vec{p}) e^{-2i \omega t} (2 \pi^3) \delta^3 (k + k') \\
            + b_r (\vec{p}) b_s^{\dagger} (\vec{p}) v_r^{\dagger} (\vec{p}) v_s (\vec{p}) (2 \pi^3) \delta^3 (k - k') \\
            + a_r^{\dagger} (\vec{p}) a_s (\vec{p}) u_r^{\dagger} (\vec{p}) u_s (\vec{p}) (2 \pi^3) \delta^3 (k - k') \\
            a_r^{\dagger} (\vec{p}) b_s^{\dagger} (\vec{p}) u_r^{\dagger} (\vec{p}) v_s (\vec{p}) e^{-2i \omega t} (2 \pi^3) \delta^3 (k + k')] \frac{d^{3} p}{(2 \pi)^3} \frac{d^{3} p'}{(2 \pi)^3} d^3 x
        \end{aligned}
    \end{equation}

    \begin{equation}
        - \gamma^0 v_r (\vec{p}) = v_r (- \vec{p}) \qquad u_s (- \vec{p}) = u_s (\vec{p})
    \end{equation}

    \begin{equation}
        \begin{aligned}
            \vec{P} = i \sum_{s} \int \frac{m}{\omega} \\
        [b_r (\vec{p}) a_s (\vec{p}) v_r^{\dagger} (\vec{p}) u_s^{\dagger} e^{-2i \omega t} (2 \pi^3) (\vec{p}) \\
        + b_r (\vec{p}) b_s^{\dagger} (\vec{p}) v_r^{\dagger} (\vec{p}) v_s (\vec{p}) \\
        + a_r^{\dagger} (\vec{p}) a_s (\vec{p}) u_r^{\dagger} (\vec{p}) u_s (\vec{p}) \\
        a_r^{\dagger} (\vec{p}) b_s^{\dagger} (\vec{p}) u_r^{\dagger} (\vec{p}) v_s (\vec{p}) e^{-2i \omega t} (2 \pi^3)] \\
        \frac{d^{3} p}{(2 \pi)^3} d^3 x
        \end{aligned}
    \end{equation}

    \begin{equation}
        \overline{v}_r (\vec{p}) u_s (\vec{p}) = 0 \qquad \overline{u}_r (\vec{p}) v_s (\vec{p}) = 0
    \end{equation}

    \begin{equation}
        \begin{aligned}
            \vec{P} = i \sum_{s} \int \frac{m}{\omega} [b_r (\vec{p}) b_s^{\dagger} (\vec{p}) v_r^{\dagger} (\vec{p}) v_s (\vec{p}) \\
        + a_r^{\dagger} (\vec{p}) a_s (\vec{p}) u_r^{\dagger} (\vec{p}) u_s (\vec{p})] \frac{d^{3} p}{(2 \pi)^3} d^3 x
        \end{aligned}
    \end{equation}

    \begin{equation}
        u_r^{\dagger} (\vec{p}) u_s (\vec{p}) = \frac{\omega}{m} \delta_{rs} \qquad v_r^{\dagger} (\vec{p}) v_s (\vec{p}) = \frac{\omega}{m} \delta_{rs}
    \end{equation}

    \begin{equation}
        \begin{aligned}
            H = i \sum_{s} \int \frac{m}{\omega} [b_r (\vec{p}) b_s^{\dagger} (\vec{p}) \frac{\omega}{m} \delta_{rs} \\
        + a_r^{\dagger} (\vec{p}) a_s (\vec{p}) \frac{\omega}{m} \delta_{rs}] \frac{d^{3} p}{(2 \pi)^3} d^3 x
        \end{aligned}
    \end{equation}

    \begin{equation}
        \begin{aligned}
            H = i \sum_{s} \int \frac{m}{\omega} [b_r (\vec{p}) b_s^{\dagger} (\vec{p}) \\
        + a_r^{\dagger} (\vec{p}) a_s (\vec{p})] \frac{d^{3} p}{(2 \pi)^3} d^3 x
        \end{aligned}
    \end{equation}

    Normal ordering then gives the final answer:

    \begin{framed}
        \begin{equation}
            \begin{aligned}
                H = i \sum_{s} \int \frac{m}{\omega} [b_r (\vec{p}) b_s^{\dagger} (\vec{p}) \\
            + a_r^{\dagger} (\vec{p}) a_s (\vec{p})] \frac{d^{3} p}{(2 \pi)^3} d^3 x
            \end{aligned}
        \end{equation}
    \end{framed}

    We can see now that normal ordering uniquely bypasses the divergence to return the empirically correct result. So we take the normal
    ordered momentum as the final result. Now that the momentum is done, let's express the charge introduced at the beginning in terms of
    the Fourier coefficients:

    \begin{equation}
        Q = \int \psi^{\dagger} \psi d^3 x
    \end{equation}

    Now we can insert the values of the fields:

    \begin{equation}
        \partial_{0} \psi (x, t) = \sum_{s} \int \frac{m}{\omega} [a_s (\vec{p}) u_s (\vec{p}) e^{- i p \cdot x} + b_s^{\dagger} (\vec{p}) v_s (\vec{p}) e^{i p \cdot x}] \frac{d^{3} p}{(2 \pi)^3}
    \end{equation}

    \begin{equation}
        \psi^{\dagger} (x, t) = \sum_{s} \int \frac{m}{\omega} [b_s (\vec{p}) v_s^{\dagger} (\vec{p}) e^{- i p \cdot x} + a_s^{\dagger} (\vec{p}) u_s^{\dagger} (\vec{p}) e^{i p \cdot x}] \frac{d^{3} p}{(2 \pi)^3}
    \end{equation}

    Inserting these gives:

    \begin{equation}
        Q = i \sum_{s} \int \frac{m}{\omega} [a_s (\vec{p}) u_s (\vec{p}) e^{- i p \cdot x} + b_s^{\dagger} (\vec{p}) v_s (\vec{p}) e^{i p \cdot x}] [b_s (\vec{p}) v_s^{\dagger} (\vec{p}) e^{- i p \cdot x} + a_s^{\dagger} (\vec{p}) u_s^{\dagger} (\vec{p}) e^{i p \cdot x}] \frac{d^{3} p}{(2 \pi)^3} d^3 x
    \end{equation}


    \begin{equation}
        \begin{aligned}
            Q = i \sum_{s} \int \frac{m}{\omega} \\
        [b_r (\vec{p}) a_s (\vec{p}) v_r^{\dagger} (\vec{p}) u_s^{\dagger} (\vec{p}) e^{- i p \cdot x} \\
        + b_r (\vec{p}) b_s^{\dagger} (\vec{p}) v_r^{\dagger} (\vec{p}) v_s (\vec{p}) e^{i p \cdot x} \\
        + a_r^{\dagger} (\vec{p}) a_s (\vec{p}) u_r^{\dagger} (\vec{p}) u_s (\vec{p}) e^{- i p \cdot x} \\
        a_r^{\dagger} (\vec{p}) b_s^{\dagger} (\vec{p}) u_r^{\dagger} (\vec{p}) v_s (\vec{p}) e^{i p \cdot x}] \\
        \frac{d^{3} p}{(2 \pi)^3} d^3 x
        \end{aligned}
    \end{equation}

    Now, the following relations can be used to do the integration over x:

    \begin{equation}
        \int e^{-i (k-k') \cdot x} d^3 x = \int e^{-i (\omega-\omega') \cdot x} e^{-i (k-k') \cdot x} d^3 x = e^{-i (\omega-\omega') \cdot x} (2 \pi^3) \delta^3 (k - k') = (2 \pi^3) \delta^3 (k - k')
    \end{equation}

    \begin{equation}
        \int e^{-i (k-k') \cdot x} d^3 x = \int e^{-i (\omega-\omega') \cdot x} e^{-i (k-k') \cdot x} d^3 x = e^{-i (\omega-\omega') \cdot x} (2 \pi^3) \delta^3 (k + k') = e^{-2i \omega t} (2 \pi^3) \delta^3 (k + k')
    \end{equation}

    Applying these:

    \begin{equation}
        \begin{aligned}
            Q = - i \int \sum_{s} \int \frac{m}{\omega} [b_r (\vec{p}) a_s (\vec{p}) v_r^{\dagger} (\vec{p}) u_s^{\dagger} (\vec{p}) e^{-2i \omega t} (2 \pi^3) \delta^3 (k + k') \\
            + b_r (\vec{p}) b_s^{\dagger} (\vec{p}) v_r^{\dagger} (\vec{p}) v_s (\vec{p}) (2 \pi^3) \delta^3 (k - k') \\
            + a_r^{\dagger} (\vec{p}) a_s (\vec{p}) u_r^{\dagger} (\vec{p}) u_s (\vec{p}) (2 \pi^3) \delta^3 (k - k') \\
            a_r^{\dagger} (\vec{p}) b_s^{\dagger} (\vec{p}) u_r^{\dagger} (\vec{p}) v_s (\vec{p}) e^{-2i \omega t} (2 \pi^3) \delta^3 (k + k')] \frac{d^{3} p}{(2 \pi)^3} \frac{d^{3} p'}{(2 \pi)^3} d^3 x
        \end{aligned}
    \end{equation}

    Now the p' integration can be done:

    \begin{equation}
        - \gamma^0 v_r (\vec{p}) = v_r (- \vec{p}) \qquad u_s (- \vec{p}) = u_s (\vec{p})
    \end{equation}

    \begin{equation}
        \begin{aligned}
            Q = i \sum_{s} \int \frac{m}{\omega} \\
        [b_r (\vec{p}) a_s (\vec{p}) v_r^{\dagger} (\vec{p}) u_s^{\dagger} e^{-2i \omega t} (2 \pi^3) (\vec{p}) \\
        + b_r (\vec{p}) b_s^{\dagger} (\vec{p}) v_r^{\dagger} (\vec{p}) v_s (\vec{p}) \\
        + a_r^{\dagger} (\vec{p}) a_s (\vec{p}) u_r^{\dagger} (\vec{p}) u_s (\vec{p}) \\
        a_r^{\dagger} (\vec{p}) b_s^{\dagger} (\vec{p}) u_r^{\dagger} (\vec{p}) v_s (\vec{p}) e^{-2i \omega t} (2 \pi^3)] \\
        \frac{d^{3} p}{(2 \pi)^3} d^3 x
        \end{aligned}
    \end{equation}

    \begin{equation}
        \overline{v}_r (\vec{p}) u_s (\vec{p}) = 0 \qquad \overline{u}_r (\vec{p}) v_s (\vec{p}) = 0
    \end{equation}

    \begin{equation}
        \begin{aligned}
            Q = i \sum_{s} \int \frac{m}{\omega} [b_r (\vec{p}) b_s^{\dagger} (\vec{p}) v_r^{\dagger} (\vec{p}) v_s (\vec{p}) \\
        + a_r^{\dagger} (\vec{p}) a_s (\vec{p}) u_r^{\dagger} (\vec{p}) u_s (\vec{p})] \frac{d^{3} p}{(2 \pi)^3} d^3 x
        \end{aligned}
    \end{equation}

    This can be reexpressed using $\{ b_s (\vec{p}), b_s^{\dagger} (\vec{p}) \} = (2 \pi)^3 \frac{\omega}{m} \delta^3 (\vec{p} - \vec{p'}) \delta_{rs}$ as follows:

    \begin{equation}
        \begin{aligned}
            Q = i \sum_{s} \int \frac{m}{\omega} [b_r (\vec{p}) b_s^{\dagger} (\vec{p}) \frac{\omega}{m} \delta_{rs} \\
        + a_r^{\dagger} (\vec{p}) a_s (\vec{p}) \frac{\omega}{m} \delta_{rs}] \frac{d^{3} p}{(2 \pi)^3} d^3 x
        \end{aligned}
    \end{equation}

    We see now that normal ordering uniquely bypasses the divergence and returns the empirically correct result. So, we take the normal ordered charge as the final
    result:

    \begin{framed}
        \begin{equation}
            Q = i \sum_{s} \int \frac{m}{\omega} [b_r (\vec{p}) b_s^{\dagger} (\vec{p})
            + a_r^{\dagger} (\vec{p}) a_s (\vec{p})] \frac{d^{3} p}{(2 \pi)^3} d^3 x
        \end{equation}
    \end{framed}

    Now let's find the meaning of the various Fourier coefficient operators by exploring their effect on energy eigenstates.s

    \begin{equation}
        H_n | E \rangle = E | E \rangle
    \end{equation}

    \begin{equation}
        \begin{aligned}
            H_n a_{s} (\vec{p}) | E = \sum_{s} \int \frac{m}{\omega} [b_s^{\dagger} (\vec{p}) b_r (\vec{p}) \\
        + a_r^{\dagger} (\vec{p}) a_s (\vec{p})] \frac{d^{3} p}{(2 \pi)^3} d^3 x
        \end{aligned}
    \end{equation}

    \begin{equation}
        \begin{aligned}
            H_n a_{s} (\vec{p}) | E = \sum_{s} \int \frac{m}{\omega} b_s^{\dagger} (\vec{p}) b_r (\vec{p}) \frac{d^{3} p}{(2 \pi)^3} d^3 x \\
        + \sum_{s} \int \frac{m}{\omega} a_r^{\dagger} (\vec{p}) a_s (\vec{p}) \frac{d^{3} p}{(2 \pi)^3} d^3 x
        \end{aligned}
    \end{equation}

    \begin{equation}
        \begin{aligned}
            H_n a_{s} (\vec{p}) | E = \sum_{s} \int \frac{m}{\omega} b_s^{\dagger} (\vec{p}) b_r (\vec{p}) \frac{d^{3} p}{(2 \pi)^3} d^3 x \\
        + \sum_{s} \int \frac{m}{\omega} [ a_r^{\dagger} (\vec{p}) a_s (\vec{p}) \{ a_r^{\dagger} (\vec{p}), a_s (\vec{p}) \} ] \frac{d^{3} p}{(2 \pi)^3} d^3 x
        \end{aligned}
    \end{equation}

    \begin{equation}
        \begin{aligned}
            H_n a_{s} (\vec{p}) | E = \sum_{s} \int \frac{m}{\omega} b_s^{\dagger} (\vec{p}) b_r (\vec{p}) \frac{d^{3} p}{(2 \pi)^3} d^3 x \\
        + \sum_{s} \int \frac{m}{\omega} [ a_r^{\dagger} (\vec{p}) a_s (\vec{p}) \frac{\omega}{m} \delta^3 (\vec{p} - \vec{p'}) \delta_{rs} ] \frac{d^{3} p}{(2 \pi)^3} d^3 x
        \end{aligned}
    \end{equation}

    \begin{framed}
        \begin{equation}
            H_n a_s (\vec{p}) = (E - \omega) a_s (\vec{p}) | E \rangle
        \end{equation}
    \end{framed}

    A virtually identical calculation gives the following result for $b_s (\vec{p})$

    \begin{framed}
        \begin{equation}
            H_n b_s (\vec{p}) = (E - \omega) b_s (\vec{p}) | E \rangle
        \end{equation}
    \end{framed}

    Therefore, we see that $a_s (\vec{p})$ and $b_s (\vec{p})$ are lowering or annihilation operators (as mentioned above). They remove a quantum of energy from the state
    they act on. Now let's characterize $a_s^{\dagger} (\vec{p})$:

    \begin{equation}
        \begin{aligned}
            H_n a_{s} (\vec{p}) | E = \sum_{s} \int \frac{m}{\omega} [b_s^{\dagger} (\vec{p}) b_r (\vec{p}) \\
        + a_r^{\dagger} (\vec{p}) a_s (\vec{p})] \frac{d^{3} p}{(2 \pi)^3} d^3 x
        \end{aligned}
    \end{equation}

    \begin{equation}
        \begin{aligned}
            H_n a_{s} (\vec{p}) | E = \sum_{s} \int \frac{m}{\omega} b_s^{\dagger} (\vec{p}) b_r (\vec{p}) \frac{d^{3} p}{(2 \pi)^3} d^3 x \\
        + \sum_{s} \int \frac{m}{\omega} a_r^{\dagger} (\vec{p}) a_s (\vec{p}) \frac{d^{3} p}{(2 \pi)^3} d^3 x
        \end{aligned}
    \end{equation}

    \begin{equation}
        \begin{aligned}
            H_n a_{s} (\vec{p}) | E = \sum_{s} \int \frac{m}{\omega} b_s^{\dagger} (\vec{p}) b_r (\vec{p}) \frac{d^{3} p}{(2 \pi)^3} d^3 x \\
        + \sum_{s} \int \frac{m}{\omega} [ a_r^{\dagger} (\vec{p}) a_s (\vec{p}) \{ a_r^{\dagger} (\vec{p}), a_s (\vec{p}) \} ] \frac{d^{3} p}{(2 \pi)^3} d^3 x
        \end{aligned}
    \end{equation}

    \begin{equation}
        \begin{aligned}
            H_n a_{s} (\vec{p}) | E = \sum_{s} \int \frac{m}{\omega} b_s^{\dagger} (\vec{p}) b_r (\vec{p}) \frac{d^{3} p}{(2 \pi)^3} d^3 x \\
        + \sum_{s} \int \frac{m}{\omega} [ a_r^{\dagger} (\vec{p}) a_s (\vec{p}) \frac{\omega}{m} \delta^3 (\vec{p} - \vec{p'}) \delta_{rs} ] \frac{d^{3} p}{(2 \pi)^3} d^3 x
        \end{aligned}
    \end{equation}

    \begin{framed}
        \begin{equation}
            H_n a_s (\vec{p}) = (E + \omega) a_s (\vec{p}) | E \rangle
        \end{equation}
    \end{framed}

    A virtually identical calculation gives the following result for $b_s (\vec{p})$

    \begin{framed}
        \begin{equation}
            H_n b_s (\vec{p}) = (E + \omega) b_s (\vec{p}) | E \rangle
        \end{equation}
    \end{framed}

    Therefore, we see that $a_s^{\dagger} (\vec{p})$ and $b_s^{\dagger} (\vec{p})$ are lowering or annihilation operators (as mentioned above). They remove a quantum of energy from the state
    they act on. From here, a few important realizations can be made. First, because $H_n$ is an integral over $a_s^{\dagger} (\vec{p}) a_s (\vec{p})$ and $b_s^{\dagger} (\vec{p}) b_s (\vec{p})$,
    its expectation values are nonnegative:

    \begin{equation}
        \langle \psi | H_n | \psi \rangle
    \end{equation}

    With this in mind, because $a_s (\vec{p})$ and $b_s (\vec{p})$ lower the value of the eigenvalue by one quantum, there must be a lowest state that gets zeroes by $a_s (\vec{p})$ and $b_s (\vec{p})$:

    \begin{equation}
        a_s (\vec{p}) | 0 \rangle = 0 \qquad b_s (\vec{p}) | 0 \rangle = 0
    \end{equation}

    This ground state will be taken as the definition of a vacuum, because it has no quanta of left to be annihilated. One can then interpret these energy quanta to be particles (in this case, spin half
    massive particles of two different types because we are quantizing a massive complex Dirac spinor field theory). With this interpretation, one can see that field quantization has had the folowing
    effect. At a given frequency. BAsically the states of the theory simply correspond to a vacuum containing some number of massive spin zero particles with various frequencies. One can act on the vacuum
    state with the creation operator to populate it with particles. A particle state can be written in the folowing way:

    \begin{equation}
        \begin{aligned}
            a_s^{\dagger} (\vec{k}) | 0 \rangle = | \vec{k}, s \rangle \\
            b_s^{\dagger} (\vec{k}) | 0 \rangle = | \vec{k}, s \rangle
        \end{aligned}
    \end{equation}

    The complete spectrum of energy eigenstates can be written as follows:

    \begin{equation}
        \prod_{i = 1}^{K} a_s^{\dagger} (\vec{k}) \prod_{j = 1}^{L} b_s^{\dagger} (\vec{k}) | 0 \rangle = | (\vec{p}_1, s_1)_a ... (\vec{p}_K, s_K)_a ... (\vec{p}_1, s_1)_b ... (\vec{p}_K, s_K)_b \rangle
    \end{equation}

    Because the creation operators anticommute with each other, these field quanta obey Fermi-Dirac statistics, and are Fermions, which obey the Pauli Exclusion Principle. The zero point on the energy scale
    is set such that the energy of the vacuum has the value zero. As a consequence:

    \begin{equation}
        \begin{aligned}
            a_s^{\dagger} (\vec{k}) | 0 \rangle = | \vec{k}, s \rangle \\
            b_s^{\dagger} (\vec{k}) | 0 \rangle = | \vec{k}, s \rangle
        \end{aligned}
    \end{equation}

    The zero point on the energy ise set such that the energy of the vacuum has the vaue zero. As a consequence:

    \begin{equation}
        \begin{aligned}
            \vec{P} | \vec{k}, s \rangle = \omega | \vec{k}, s \rangle \\
            \vec{P} | \vec{k}, s \rangle = \omega | \vec{k}, s \rangle
        \end{aligned}
    \end{equation}

    With the above calculated formula for the momentum operator, one finds:

    \begin{equation}
        \begin{aligned}
            \vec{P} | \vec{k}, s \rangle = \vec{k} | \vec{k}, s \rangle \\
            \vec{P} | \vec{k}, s \rangle = \vec{k} | \vec{k}, s \rangle
        \end{aligned}
    \end{equation}

    Now for the key result from this charge calculation. Applying the charge operator to the two different types of particle states gives the following relations:

    \begin{equation}
        \begin{aligned}
            Q a^{\dagger} (\vec{k}) | 0 \rangle = a^{\dagger} (\vec{k}) | 0 \rangle \\
            Q b^{\dagger} (\vec{k}) | 0 \rangle = b^{\dagger} (\vec{k}) | 0 \rangle
        \end{aligned}
    \end{equation}

    This cements our interpretation of these two different types of particle states as particles and their antiparticles.

\end{document}